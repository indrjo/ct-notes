
\section{Definition}

\NotaInterna{Expand this section with details about duality and \TeX{}
  some initial exercises.}

\begin{definition}[Limits \& colimits]\label{definition:LimitsAndColimits}
  Let \(\cat I\) and \(\cat C\) be two categories. For every object
  \(v\) of \(\cat C\) we have the {\em constant functor}
  \[k_v : \cat I \to \cat C\] where \(k_v(i) := v\) for every object
  \(i\) and \(k_v(f) := \id_v\) for every morphism
  \(f\). \NotaInterna{Later, in the chapter of the adjunctions, we
    will introduce one functor
    \(\Delta : \cat C \to [\cat I, \cat C]\).} A {\em limit} of a functor
  \(F : \cat I \to \cat C\) is some object \(v\) of \(\cat C\) with a
  natural transformation \(\lambda : k_v \tto F\) such that:
  % \begin{quotation}
  for any object \(a\) of \(\cat C\) and \(\mu : k_a \tto F\) there is
  one and only one \(f : a \to v\) of \(\cat C\) such that
  \[\begin{tikzcd}[row sep=tiny]
      a \ar["{\mu_i}", dr] \ar["f", dd, swap] \\
      & F(i) \\
      v \ar["{\lambda_i}", ur, swap]
    \end{tikzcd}\] commutes for every \(i\) in \(\cat I\).
  % \end{quotation}
  A {\em colimit}, instead, is an object \(u\) of \(\cat C\) together
  with a \(\chi : F \tto k_u\) that has the property:
  % \begin{quotation}
  for every object \(b\) of \(\cat C\) and \(\xi : F \tto k_b\) there
  exists one and only one \(g : u \to b\) of \(\cat C\) that makes
  \[\begin{tikzcd}[row sep=tiny]
      u  \ar["g", dd, swap] \\
      & F(i) \ar["{\chi_i}", ul, swap] \ar["{\xi_i}", dl]\\
      b
    \end{tikzcd}\] commute for every \(i\) in \(\cat I\).
  % \end{quotation}
\end{definition}

% In general, (co)cones can be twisted things and (co)limits, if
% exist, can be even more so. Yes, for a functor
% \(\cat I \to \cat C\), the category \(\cat C\) has its share, but it
% is \(\cat I\) who has the last say in the research of
% (co)limits. The role of \(\cat I\) is to give a \q{shape} of the
% limits we are looking for, indeed.

\begin{example}
  We have already seen how a preordered set is a category; in this
  example let us employ \(\naturals\) with the usual ordering
  \(\le\). First of all, let us figure out what cones and cocones of
  functors \(H : \naturals \to \cat C\) are. Such functors, in other
  words, are sequence of objects and morphisms of \(\cat C\) so
  arranged
  \[H_0 \functo{\partial_0} H_1 \functo{\partial_2} \cdots \functo{\partial_{n-1}} H_n
    \functo{\partial_n} H_{n+1} \functo{\partial_{n+1}} \cdots\] In this case, a cone on
  \(H\) is a collection
  \(\set{\alpha_i : A \to H_i \mid i \in \naturals}\) such that
  \(\alpha_j = \partial_{j-1} \cdots \partial_i \alpha_i\) for every
  \(i, j \in \naturals\) such that \(i < j\); remember that if
  \(j > i\) there is no morphism \(H_j \to H_i\) and, for
  \(i \in \obj{\cat I}\), the morphism \(H_i \to H_i\) is the identity.
  \NotaInterna{Continue after you have fixed some parts before.}
  % \newline Let us now say what limits and colimits of
  % \(H : \naturals \to \cat C\) are.
\end{example}

\begin{example}
  \NotaInterna{Rewrite.} Let \(\cat C\) be a category and
  \(\mathbf 1\) a category that has one object and one morphism, and
  take a functor \(f : \mathbf 1 \to \cat C\), some \(v \in \cat C\) and
  the corresponding constant functor \(k_v : \mathbf 1 \to \cat C\). A
  natural transformation \(\zeta : k_v \tto f\) amounts of a single
  morphism \(v \to \tilde f\) of \(\cat C\), where \(\tilde f\)
  indicates the image of the unique object of \(\mathbf 1\) via
  \(f\). Thus, a limit of \(f\) is some \(v \in \obj{\cat C}\) and a
  morphism \(\lambda : v \to \tilde f\) of \(\cat C\) such that: for every
  object \(u\) and morphism \(\mu : u \to \tilde f\) in \(\cat C\), there
  is a unique morphism \(u \to v\) of \(\cat C\) that makes
  \[\begin{tikzcd}[row sep=tiny]
      u \ar["\mu", dr] \ar[dd] \\
      & \tilde f \\
      v \ar["\lambda", ur, swap]
    \end{tikzcd}\] commute.
\end{example}

\begin{exercise}
  What are colimts of functors \(\mathbf 1 \to \cat C\)?
\end{exercise}

\begin{example}
  \NotaInterna{Rewrite.} Consider a monoid (viz a single object
  category) \(\cat G\): for the scope of this example we write \(G\)
  for the set of the morphisms of \(\cat G\). Let
  \(F : \cat G \to \Set\) be a functor, and let \(\hat F\) indicate the
  \(F\)-image of the unique object of \(\cat G\) whilst, for
  \(f \in G\), \(\hat f\) the function \(F(f) : \hat F \to \hat F\). Now,
  being \(k_X : \cat G \to \Set\) the functor constant at \(X\), with
  \(X\) a set, a natural transformation \(\lambda : F \tto k_X\) is a
  morphism \(\lambda : \hat F \to X\) such that
  \(\lambda = \lambda \hat f\) for every \(f \in G\). These two things, the set
  \(X\) and the function \(\lambda\), together are a colimit of \(F\)
  whenever
  \begin{quotation}
    for every set \(Y\) and function \(\mu : \hat F \to Y\) such that
    \(\mu = \mu \hat f\) for every \(f \in G\) there exists one and only one
    function \(h : X \to Y\) such that \(\mu = h \lambda\).
  \end{quotation}
  \NotaInterna{Is that thing even interesting?} \NotaInterna{Write
    about functors \(\cat G \to \Set\)\dots{}}
\end{example}

\noindent\NotaInterna{Write about duality here. Explain how limits and
  colimits are dual\dots{}}

The following is very basic property: limits of a same functor are are
essentially the same.

\begin{proposition}\label{proposition:LimitsAreIsomorphic}
  Let \(F : \cat I \to \cat C\) be a functor. If
  \(\set{\eta_i : a \to F(i) \mid i \in \obj{\cat I}}\) and
  \(\set{\theta_i : b \to F(i) \mid i \in \obj{\cat I}}\) are limits of
  \(F\), then \(a \cong b\).
\end{proposition}

\begin{proof}
  By definition of limit, we a have a unique \(f : a \to b\) and a
  unique \(g : b \to a\) making the triangles in
  \[\begin{tikzcd}[row sep=small]
      a \ar["{\eta_i}", dr] \ar["f" description, dd, bend left] \\
      & F(i) \\
      b \ar["{\theta_i}", ur, swap] \ar["g" description, uu, bend left]
    \end{tikzcd}\] commute for every object \(i\) of \(\cat I\). In
  this case,
  \[\begin{aligned}
    & \eta_i = \theta_i f = \eta_i (gf) \\
    & \theta_i = \eta_i g = \theta_i (fg)
  \end{aligned}\]
Invoking again the universal property of limits, \(gf = \id_a\) and \(fg = \id_b\).
\end{proof}

Fortunately, there are few shapes that are both ubiquitous and
simple. This section is dedicated to them, while in the successive one
we will prove (Proposition~\ref{proposition:Completeness}) that if
some simple functors have limits, then all the functors do have
limits.

%%% Local Variables:
%%% mode: LaTeX
%%% TeX-master: "../CT"
%%% TeX-engine: luatex
%%% End:
