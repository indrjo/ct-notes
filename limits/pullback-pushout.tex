% !TEX program = lualatex
% !TEX spellcheck = en_GB
% !TEX root = ../limits.tex

\section{Pullbacks and pushouts}

Let \(\cat C\) be a category. The limits of the functors
\[\left(\begin{tikzcd}[sep=tiny]
\bullet \ar[dr] &         & \bullet \ar[dl] \\
                & \bullet &
\end{tikzcd}\right) \to \cat C\]
are called {\em pullbacks}. Dually, the colimits of the functors
\[\left(\begin{tikzcd}[sep=tiny]
        & \bullet \ar[dl] \ar[dr] &         \\
\bullet &                         & \bullet
\end{tikzcd} \right) \to \cat C\]
are said {\em pushouts}. More explicitly:

\begin{definition}[Pullbacks \& pushouts, explicit]
Let \(\cat C\) a category and a pair of morphisms with the same codomain
\begin{equation}
\begin{tikzcd}[row sep=tiny]
a \ar["f", dr, swap] &   & b \ar["g", dl] \\
                     & c &
\end{tikzcd}
\label{diagram:MorphismsToPullback}\end{equation}
in \(\cat C\). A {\em pullback} of~\eqref{diagram:MorphismsToPullback} is any pair of morphisms with a common domain
\[\begin{tikzcd}[row sep=tiny]
  & p \ar["h", dl, swap] \ar["k", dr] &   \\
a &                                   & b
\end{tikzcd}\]
in \(\cat C\) such that:
\begin{tcbitem}
\item the square
\[\begin{tikzcd}[row sep=tiny]
  & p \ar["h", dl, swap] \ar["k", dr] &   \\
a \ar["f", dr, swap] &   & b \ar["g", dl] \\
                     & c &
\end{tikzcd}\]
commutes
\item for every
\[\begin{tikzcd}[row sep=tiny]
  & p' \ar["{h'}", dl, swap] \ar["{k'}", dr] &   \\
a &                                          & b
\end{tikzcd}\]
in \(\cat C\)
%such that \(fh' = gk'\) 
making
\[\begin{tikzcd}[row sep=tiny]
  & p \ar["{h'}", dl, swap] \ar["{k'}", dr] &   \\
a \ar["f", dr, swap] &   & b \ar["g", dl] \\
                     & c &
\end{tikzcd}\]
commute there exists one and only one \(r : p' \to p\) in \(\cat C\) such that
\[\begin{tikzcd}
& p' \ar["{h'}", dl, swap] \ar["r", d, dashed] \ar["{k'}", dr] & \\
a & p \ar["h", l] \ar["k", r, swap] & b
\end{tikzcd}\]
commutes.
\end{tcbitem}
Assuming now we have a pair of morphisms with the same domain
\begin{equation}
\begin{tikzcd}[row sep=tiny]
& c \ar["f", dl, swap] \ar["g", dr] & \\
a & & b
\end{tikzcd}
\label{diagram:MorphismsToPushout}\end{equation}
in \(\cat C\), a {\em pushout} of~\eqref{diagram:MorphismsToPushout} is any pair of morphisms in \(\cat C\) with a common codomain
\[\begin{tikzcd}[row sep=tiny]
a \ar["h", dr, swap] & & b \ar["k", dl] \\
& q &
\end{tikzcd}\]
such that:
\begin{tcbitem}
\item the square
\[\begin{tikzcd}[row sep=tiny]
                     & c \ar["f", dl, swap] \ar["g", dr] &                \\
a \ar["h", dr, swap] &                                   & b \ar["k", dl] \\
                     & q                                 &
\end{tikzcd}\]
commutes
\item for every
\[\begin{tikzcd}[row sep=tiny]
a \ar["{h'}", dr, swap] & & b \ar["{k'}", dl] \\
& q' &
\end{tikzcd}\]
in \(\cat C\)
%such that \(h'f = k'g\)
making
\[\begin{tikzcd}[row sep=tiny]
                     & c \ar["f", dl, swap] \ar["g", dr] &                \\
a \ar["{h'}", dr, swap] &                                   & b \ar["{k'}", dl] \\
                     & q                                 &
\end{tikzcd}\]
commute there exists one and only one \(s : q \to q'\) in \(\cat C\) such that
\[\begin{tikzcd}
a \ar["h", r] \ar["{h'}", dr, swap] & q \ar["s", d, dashed] & b \ar["k", l, swap] \ar["{k'}", dl] \\
& q' &
\end{tikzcd}\]
commutes.
\end{tcbitem}
\end{definition}

The definitions above become can be more concise though: pullbacks (pushouts) are products (coproducts) in certain categories.

\begin{proposition}
A pullback of
\[\begin{tikzcd}[row sep=tiny]
a \ar["f", dr, swap] &   & b \ar["g", dl] \\
                     & c &
\end{tikzcd}\]
in \(\cat C\) is any of the products that pair of morphisms in \(\cat C {\downarrow} c\). Dually, a pushout of
\[\begin{tikzcd}[row sep=tiny]
& c \ar["f", dl, swap] \ar["g", dr] & \\
a & & b
\end{tikzcd}\]
in \(\cat C\) is any of the coproducts such pair of morphisms in \(c {\downarrow} \cat C\).
\end{proposition}

\begin{proof}
This is \inlinethm{exercise}.
\end{proof}

\begin{exercise}
Let \(\cat C\) be a category with initial object \(0\) and terminal object \(1\). What are pullbacks of a pair of morphisms with codomain \(1\)? What are pushouts of a pair of morphisms with domain \(0\)?
\end{exercise}

\begin{example}[Pullbacks in \(\Set\)]\label{example:PullbacksInSet}
Now we consider sets and functions as in
\[\begin{tikzcd}
                    & A \ar["f", d] \\
B \ar["g", r, swap] & C
\end{tikzcd}\]
with the aim to find a pullback of it. From the set
\[D := \set{(a, b) \in A \times B \mid f(a) = g(b)}\]
we can make the functions
\begin{align*}
& p : D \to A\,, \ p(a, b) := a \\
& q : D \to B\,, \ q(a, b) := b
\end{align*}
Hence we can draw at least the commuting square
\[\begin{tikzcd}
D \ar["p", r] \ar["q", d, swap] & A \ar["f", d] \\
B \ar["g", r, swap]             & C
\end{tikzcd}\]
Now consider
\[\begin{tikzcd}
D' \ar["{p'}", drr, bend left] \ar["{q'}", ddr, bend right, swap]                 \\
& D \ar["p", r] \ar["q", d, swap]                                & A \ar["f", d] \\
& B \ar["g", r, swap]                                            & C
\end{tikzcd}\]
with \(fp' = gq'\). This hypothesis implies that \(\left(p'(x), q'(x)\right) \in D\) for every \(x \in D'\), and allows us to introduce the function
\[r : D' \to D\,, \ r(x) := \left(p'(x), q'(x)\right)\]
which is such that \(p r = p'\) and \(q r = q'\). Finally, \(r\) is the unique one to do so, which fact is immediate for how \(r\) is defined.
\end{example}

Let us remain in \(\Set\). Consider a function \(f : A \to B\) and the diagram
\[\begin{tikzcd}
                    & A \ar["f", d] \\
A \ar["f", r, swap] & B
\end{tikzcd}\]
where we have duplicated \(f\). The example above tells us we have the pullback square
\[\begin{tikzcd}
R_f \ar["p", r] \ar["q", d, swap] & A \ar["f", d] \\
A \ar["f", r, swap]               & B
\end{tikzcd}\]
with \(R_f := \set{(a, b) \in A \times A \mid f(a) = f(b)}\). This subset of \(A \times A\) is a certain equivalence relation over \(A\), namely the {\em kernel relation} of \(f\). \NotaInterna{Did we mention kernel relations in the intro?}

There is nothing special of \(\Set\) that prevents us to generalize it to any category \(\cat C\): we define the {\em kernel relation} of a \(f : a \to b\) in \(\cat C\) to be any of the pullbacks of
\[\begin{tikzcd}
                    & a \ar["f", d] \\
a \ar["f", r, swap] & b
\end{tikzcd}\]

As soon as we meet coequalizers, we will have the tool to express the quotient \({X}{/}{R_f}\) in a categorial fashion, and thus to motivate the general concept of {\em quotient object}.

\begin{example}[Pushouts in \(\Set\)]\label{example:PushoutsInSet}
Recall what we have done in Example~\ref{example:CoprodOfSets}, but change a bit the notation. Take a family of two sets \(A_1\) and \(A_2\): write \(A_1+A_1\) instead of using the \(\sum\) or \(\coprod\) notation, write \(\linj\) and \(\rinj\) in place of \(\inj_1\) and \(\inj_2\), respectively.\newline
To get started, let us consider sets and functions
\[\begin{tikzcd}[row sep=tiny]
  & C \ar["f", dl, swap] \ar["g", dr] &   \\
A &                                   & B 
\end{tikzcd}\]
Let us draw a diagram
\begin{equation}\begin{tikzcd}
C \ar["f", r] \ar["g", d, swap]  & A \ar["\linj", d] \\
B \ar["\rinj", r, swap] & A+B
\end{tikzcd}\label{diagram:NotYetACommSquareInSet}\end{equation}
By definition of \(A+B\), it can be \(\mathtt{left} f(x) \ne \mathtt{right} g(x)\) for some \(x \in C\). However, we can make them \q{equal} under an adequate equivalence relation \(\sim\): the smallest in which, for \(x \in C\), the elements \(\linj f(x)\) and \(\rinj g(x)\) are identified; that is we define \(\sim\) to be the smallest equivalence relation containing
\[R := \set{(\linj f(x), \rinj g(x)) \mid x \in C}.\]
In this case, let us write \(p\) the projection \(A+B \to \frac{A+B}{\sim}\). The new square is
\begin{equation}\begin{tikzcd}
C \ar["f", r] \ar["g", d, swap] & A \ar["h", d]    \\
B \ar["k", r, swap]             & \frac{A+B}{\sim}
\end{tikzcd}\label{diagram:NowACommSquareInSet}\end{equation}
with \(h := p \linj\) and \(k := p \rinj\) and it is commutative. Now, pick
\[\begin{tikzcd}
C \ar["f", r] \ar["g", d, swap] & A \ar["h", d] \ar["{h'}", ddr, bend left] \\
B \ar["k", r, swap] \ar["{k'}", drr, bend right, swap] & \frac{A+B}{\sim} \\
 & & X
\end{tikzcd}\]
such that \(h'f = k'g\). By the universal property of coproduct, we have the function
\[h' + k' : A + B \to X\]
such that \(\left(h' + k'\right) \linj =  h'\) and \(\left(h' + k'\right) \rinj =  k'\). Taken, for any \(x \in C\), one \((\linj f(x), \rinj g(x)) \in R\), we have
\begin{align*}
& (h'+g')(\rinj f(x)) = h'f(x) \\
& (h'+g')(\linj g(x)) = k'g(x) 
\end{align*}
which are equal for every \(x \in C\), by assumption. \NotaInterna{Talk about generated equivalence relations and what follows.} Thus the triangle
\[\begin{tikzcd}[column sep=tiny]
A+B \ar["{h'+k'}", rr] \ar["p", dr, swap] & & X \\
 & \frac{A+B}{\sim} \ar[ur, dashed, swap]
\end{tikzcd}\]
commutes for exactly one dashed function. This function is the one we are looking for. \NotaInterna{Complete\dots{}}
\end{example}

\begin{exercise}
In the previous example, what is \(\frac{A+B}{\sim}\) if \(C = A \cap B\) and \(f\) and \(g\) are just the inclusions of \(C\) in \(A\) and \(B\) respectively? \NotaInterna{There is other material to put here\dots{}}
\end{exercise}

\begin{exercise}
Go back to Example~\ref{example:PushoutsInSet}. What if we started our discourse from
\[\begin{tikzcd}
C \ar["f", r] \ar["g", d, swap]  & A \ar[d, hookrightarrow] \\
B \ar[r, hookrightarrow]         & A \cup B
\end{tikzcd}\]
instead?
%You may tackle this question from two different points: trying to readapt the discourse or prove that \((A \cup B){/}{\sim'} \cong (A+B){/}{\sim}\), where \(\sim'\) is the equivalence relation generated by
%\[f(x) \sim' g(x) \ \text{for } x \in C.\] 
\end{exercise}

%\begin{lemma}
%Let \(\cat C\) be a category and
%\begin{equation}\begin{tikzcd}[row sep=tiny]
%a \ar["f", dr, swap] &   & b \ar["g", dl] \\
%                     & c &
%\end{tikzcd}\label{diagram:MorphismsToBePullbacked}\end{equation}
%a couple of morphisms in \(\cat C\). If \(\cat C\) has a product
%\begin{equation}\begin{tikzcd}[row sep=tiny]
%  & p \ar["h", dl, swap] \ar["k", dr] &   \\
%a &                                   & b
%\end{tikzcd}\label{diagram:ProductToBePullback}\end{equation}
%of \(a\) and \(b\) such that the square
%\[\begin{tikzcd}[row sep=tiny]
%  & p \ar["h", dl, swap] \ar["k", dr] &   \\
%a \ar["f", dr, swap] &   & b \ar["g", dl] \\
%                     & c &
%\end{tikzcd}\]
%commutes, then~\eqref{diagram:ProductToBePullback} is a pullback of~\eqref{diagram:MorphismsToBePullbacked}.
%\end{lemma}

\begin{exercise}[Gluing topological spaces]
If you are given two spaces \(X\) and \(A\), a subspace \(E \subseteq A\) and a continuous function \(f : E \to X\), then \(X \sqcup_f A\) denotes the disjoint union \(X \sqcup A\) where every \(x \in E\) is identified to \(f(x)\), that is% . More formally: on \(X \sqcup A\) we define the equivalence relation \(\sim_f\) as the smallest one containing \(R_f := \set{(f(x), x) \mid x \in E}\) and
\[X \sqcup_f A := \frac{X \sqcup A}{x \sim f(x) \text{ for } x \in E} .\]
Find a pushout square
\[\begin{tikzcd}
E \ar[r, hookrightarrow] \ar["f", d, swap] & A \ar[d]     \\
X \ar[r]                                   & X \sqcup_f A
\end{tikzcd}\]
in \(\Top\). The exercise requires you to work about the topologies involved and about continuity.
%with the arrow \(\hookrightarrow\) indicating a mere inclusion. Does this look familiar? Yes! Continue. Part of the exercise is to reason about the topology over the stuff involved here.
\end{exercise}

\begin{example}[CW complexes]
In Topology, several spaces often employed are homotopic --- or even homeomorphic --- to other spaces glued together. Although you can glue everything to everything, very simple spaces to attach are disks \(\mathbb D^n := \set{x \in \mathbb R^n \mid \lVert x \rVert \le 1}\) along their boundaries \(\mathbb S^{n-1} := \partial \dsc^n\). (Pay attention to superscripts.) For any topological space \(X\), we can perform the following recursive construction:
\begin{tcbitem} 
\item Let \(X_0\) be the space \(X\) but with the discrete topology.
\item For \(n \in \naturals\), from topological space \(X_n\) we prescribe the construction of another space \(X_{n+1}\). If we are given a family \(\set{D_\alpha \mid \alpha \in \Lambda}\) of copies of \(\mathbb D^{n+1}\) and collection of continuous functions
\[\set{\left. f_\alpha : \partial D_\alpha \to X_n \right\mid \alpha \in \Lambda}\]
%Define now \(\sim\) to be the smallest equivalence relation that identifies every \(x \in \partial D_\alpha^{n+1}\) with \(f_\alpha^{n+1}(x)\), for \(\alpha \in \Lambda\).
%\[X_{n+1} := \frac{X_n \sqcup \coprod_{\alpha \in \Lambda} D_\alpha^{n+1}}{\sim}\]
then we can consider the following topological space
\[X_{n+1} := \frac{X_n \sqcup \coprod_{\alpha \in \Lambda} D_\alpha}{
  x \sim f_\alpha(x) \text{ for } \alpha \in \Lambda \text{ and } x \in \partial D_\alpha
}.\]
In other words, \(X_{n+1}\) is \(X_n\) with \((n+1)\)-dimensional disks attached to it along their boundaries.
\end{tcbitem}
As always, we are striving to find some universal property worth of consideration. A square comes easily if you consider the inclusions running in parallel \(\mathbb S^n \hookrightarrow \mathbb D^{n+1}\) and \(X_n \hookrightarrow X_{n+1}\) together with \(\mathbb D^{n+1} \hookrightarrow X_{n+1}\) of the construction above. The other pieces are the attaching maps: indeed we have a commuting square
%\NotaInterna{A lot of ugly notation and imprecision here\dots{}} As before, at every step of recursive construction just described, we have a commuting square
\[\begin{tikzcd}[column sep=large]
\coprod_{\alpha \in \Lambda} \partial D_\alpha
  \ar[r, hookrightarrow]
  \ar["{\coprod_{\alpha \in \Lambda} f_\alpha}", d, swap]
&
\coprod_{\alpha \in \Lambda} D_\alpha  \ar[d, hookrightarrow]
\\
X_n \ar[r, hookrightarrow] & X_{n+1}
\end{tikzcd}\]
This square is a pushout one in \(\Top\), which is easy to prove.
\end{example}

\begin{exercise}[Spheres are CW complexes] Consider the case in which \(X_0\) is a single point space and so are the spaces \(X_i\) for \(1 \le i \le n-1\). Such situation can be achieved by attaching no disk for a while; afterwards, attach one disk \(\dsc^{n}\) along \(\sph^{n-1}\) to \(X_{n-1}\). Hence we have a homeomorphism \(X_n \cong {\dsc^n}{/}{\sph^{n-1}}\), but you will show something more, that is \(X_n \cong \sph^n\).\newline
In your Topology course, you might have managed to show this as follows:
\begin{tcbenum}
\item You have constructed a surjective continuous function \(f : \dsc^n \to \sph^n\) and considered the quotient space \({\dsc^n}{/}{\sim_f}\) where \(\sim_f\) is the kernel relation \NotaInterna{talk about kernel relations!} of \(f\). This relation is not a random relation: for every \(x, y \in \dsc^n\), we have \(x \sim_f y\) if and only if \(x = y\) or \(x, y \in \sph^{n-1}\). As consequence, \({\dsc^n}{/}{\sim_f} = {\dsc^n}{/}{\sph^{n-1}}\).
\item Thanks to the universal property of quotients, the function \(\phi : {\dsc^n}{/}{\sph^{n-1}} \to \sph^n\) such that \(f = \phi p_n\), with the \(p_n\) the canonical projection, is continuous and bijective. Now, recalling that continuous functions from compact spaces to Hausdorff spaces are closed, conclude that indeed \(\phi\) is a homeomorphism.
\end{tcbenum}
%We will construct a continuous function \(f : \dsc^n \to \sph^n\), let us see if we can recycle something.\newline
%To get the idea imagine the situation with \(n=2\). The space \(\dsc^n\) is covered by two closed spaces
%\begin{align*}
%& A := \set{x \in \dsc^n \mid \Abs x \le \frac12} \\
%& B := \set{x \in \dsc^n \mid \frac12 \le \Abs x}
%\end{align*}
%So let us define two continuous functions \(f_A : A \to \sph^2\) and \(f_B : B \to \sph^n\) such that coincide on the intersection of \(A\) and \(B\) and the resulting function
%\[f : \dsc^n \to \sph^n\,, \ f(x) := \begin{cases} f_A(x) & \text{if } x \in A \\ f_B(x) & \text{if } x \in B \end{cases}\]
%is continue and surjective. The first is simple
%\[f_A(x) := \left(2x, -\sqrt{1-4 {\Abs x}^2}\right)\]
%whereas the second requires little more time
%\[f_B(x) := \left(2\left(\frac{x}{\Abs x}-x\right), \sqrt{1-4 \Abs{\frac{x}{\Abs x}-x}^2}\right)\]
%\NotaInterna{Some pictures in preparation help to understand the idea behind...}
The aim of this exercise it that you can arrive to the same result in a different manner. If you can recollect your memories or retrieve your notes, see if you can recycle the \(f\) above and write a pushout square
\begin{equation}\begin{tikzcd}
\sph^{n-1} \ar[hookrightarrow, r] \ar["!", d, swap] & \dsc^n \ar["f", d] \\
X_{n-1} \ar[r] & \sph^n
\end{tikzcd}\label{diagram:SpherePushout}\end{equation}
If you do not know how \({\dsc^n}{/}{\sph^{n-1}} \cong \sph^n\), it does not matter since you will force yourself to search for a pushout square like~\eqref{diagram:SpherePushout}.
\end{exercise}

\begin{exercise}[\(\rpspc^n\) is a CW complex]
In Topology, the \(n\)-th {\em real projective space} is
\[\rpspc^n := \frac{\reals^{n+1} \setminus \set{0}}{x \sim \lambda x \text{ for } x \in \reals^{n+1}, \lambda \in \reals}\]
which is known to be homeomorphic to the sphere \(\sph^n\) which has the antipodal points identified:
\[\frac{\sph^n}{\displaystyle x \sim - x \text{ for } x \in \sph^n} .\]
This observation is the key for the coming arguments. In fact, \(\sph^n\) is the boundary of \(\dsc^{n+1}\) and we already have a continuous function \(p_n : \sph^n \to \rpspc^n\) that attaches the the disc to the projective space along the boundary. Find a pushout square of the form
\[\begin{tikzcd}
\sph^n \ar[hookrightarrow, r] \ar["p_n", d, swap] & \dsc^{n+1} \ar["?", d] \\
\rpspc^n \ar[r, hookrightarrow] & \rpspc^{n+1}
\end{tikzcd}.\]
%This has remarkable consequence: here is how the real projective spaces are CW complexes!
\end{exercise}

Topology, again, but combined with Group Theory.

\begin{example}[Seifert-van Kampen Theorem]
Suppose given a topological space \(X\), two open subsets \(A, B \subseteq X\) such that \(A \cup B = X\) and one point \(x_0\) of \(A \cap B\). Let us denote by \(i_A\), \(i_B\), \(j_A\) and \(j_B\) the group morphisms induced by the inclusions \(A \cap B \hookrightarrow A\), \(A \cap B \hookrightarrow B\), \(A \hookrightarrow X\) and \(B \hookrightarrow X\), respectively. If \(A\), \(B\) and \(A \cap B\) are path-connected then,
\[\begin{tikzcd}[sep=small]
 & \pi_1(A, x_0) \ar["{j_A}", dr] & \\
\pi_1(A \cap B, x_0) \ar["{i_A}", ur] \ar["{i_B}", dr, swap] & & \pi_1(X, x_0) \\
 & \pi_1(B, x_0) \ar["{j_B}", ur, swap]
\end{tikzcd}\]
is a pushout square of \(\Grp\).
\end{example}

\NotaInterna{The Pullback Lemma is dropped here without a precise plan to embed it nicely with examples and further development. It's an issue that must be fixed.}

\begin{proposition}[The Pullback Lemma]
In a category \(\cat C\) consider a diagram
\[\begin{tikzcd}
\bullet \ar["a", r] \ar["d", d, swap] \ar["{Q_1}", dr, phantom] & \bullet \ar["b", r] \ar["g" description, d] \ar["{Q_2}", dr, phantom] & \bullet \ar["c", d]\\
\bullet \ar["e", r, swap] & \bullet \ar["f", r, swap] & \bullet
\end{tikzcd}\]
where the perimetric rectangle commutes and the square on the right is a pullback one. Then that on the left is a pullback square is and only if so is the outer rectangle. 
\end{proposition}

\begin{proof}
Let us assume \(Q_1\) is a pullback square first. Consider any choice of \(h\) and \(k\) such that \(ch = f(ek)\):
\[\begin{tikzcd}
\bullet \ar["h", bend left=10, drrr, blue] \ar["k", bend right=10, ddr, swap, blue] & & & \\
& \bullet \ar["a", r, swap] \ar["d", d] \ar["{Q_1}", dr, phantom] & \bullet \ar["b", r, swap] \ar["g", d] \ar["{Q_2}", dr, phantom] & \bullet \ar["c", d, blue]\\
& \bullet \ar["e", r, swap, blue] & \bullet \ar["f", r, swap, blue] & \bullet
\end{tikzcd}\]
Being \(Q_2\) a pullback square, there exists one and only one \(l\) such that \(h = bl\) and \(gl = ek\).
\[\begin{tikzcd}
\bullet \ar["h", bend left=10, drrr, blue] \ar["k", bend right=10, ddr, swap, blue] \ar["l"{description}, bend left=8, drr] & & & \\
& \bullet \ar["a", r, swap] \ar["d", d] \ar["{Q_1}", dr, phantom] & \bullet \ar["b", r, swap] \ar["g", d] \ar["{Q_2}", dr, phantom] & \bullet \ar["c", d, blue]\\
& \bullet \ar["e", r, swap, blue] & \bullet \ar["f", r, swap, blue] & \bullet
\end{tikzcd}\]
We have just said that this square in red commutes:
\[\begin{tikzcd}
\bullet \ar["h", bend left=10, drrr] \ar["k", bend right=10, ddr, swap, red] \ar["l"{description}, drr, bend left=8, red] & & & \\
& \bullet \ar["a", r, swap] \ar["d", d] \ar["{Q_1}", dr, phantom] & \bullet \ar["b", r, swap] \ar["g", d, red] \ar["{Q_2}", dr, phantom] & \bullet \ar["c", d]\\
& \bullet \ar["e", r, swap, red] & \bullet \ar["f", r, swap] & \bullet
\end{tikzcd}\]
Now, being \(Q_1\) a pullback square, we have one \(m\) such that\(l = am\) and \(k = dm\):
\[\begin{tikzcd}
\bullet \ar["h", bend left=10, drrr] \ar["k", bend right=10, ddr, swap, red] \ar["l"{description}, drr, bend left=8, red] \ar["m"{description}, dr] & & & \\
& \bullet \ar["a", r, swap] \ar["d", d] \ar["{Q_1}", dr, phantom] & \bullet \ar["b", r, swap] \ar["g", d, red] \ar["{Q_2}", dr, phantom] & \bullet \ar["c", d]\\
& \bullet \ar["e", r, swap, red] & \bullet \ar["f", r, swap] & \bullet
\end{tikzcd}\]
At this point, we have \(b a m = b l = h\) and \(dm = k\). To conclude the first half of the theorem, you have to pick any \(m'\) making commute the triangles in green:
\[\begin{tikzcd}
\bullet \ar["h", bend left=10, drrr, green!65!black] \ar["k", bend right=10, ddr, swap, green!65!black] \ar["am" description, drr, bend left=8] \ar["{m'}" description, dr, green!65!black] & & & \\
& \bullet \ar["a", r, swap, green!65!black] \ar["d", d, green!65!black] \ar["{Q_1}", dr, phantom] & \bullet \ar["b", r, swap, green!65!black] \ar["g", d] \ar["{Q_2}", dr, phantom] & \bullet \ar["c", d]\\
& \bullet \ar["e", r, swap] & \bullet \ar["f", r, swap] & \bullet
\end{tikzcd}\]
Being \(Q_2\) a pullback square, we have \(am' = am\). In conclusion, being \(Q_1\) a pullback square too, we have \(m = m'\). \NotaInterna{Finer explanation here\dots{}}
\end{proof}

\begin{exercise}
Prove the remaining part of the theorem above.
\end{exercise}

\begin{example}[Character functions]
Consider a subset \(A\) of some larger set \(X\). You sure know a simple function called {\em character function} with just says if an element of \(X\) is a member of \(A\):
\[\chi_A : X \to \set{\mathtt{true}, \mathtt{false}}\,,\ \chi_A (x) := \begin{cases} \mathtt{true} & \text{if } x \in A \\ \mathtt{false} & \text{otherwise} \end{cases}\]
From now on, let us write \(\Omega\) to mean \(\set{\mathtt{true}, \mathtt{false}}\). As always, let us draw what we have:
\[\begin{tikzcd}
A \ar[r, hookrightarrow] & X \ar["{\chi_A}", d] \\
& \Omega
\end{tikzcd}\]
with \(A \hookrightarrow X\) being the usual inclusion. The composition of such functions is function constant to \(\mathtt{true}\). We know that constant functions are such because they can be factored through some function \(A \to 1\) \NotaInterna{write about this explicitly somewhere}, which results in a commuting square
\[\begin{tikzcd}
A \ar[r, hookrightarrow] \ar["!", d, swap] & X \ar["{\chi_A}", d] \\
1 \ar["{\lam x. \mathtt{true}}", r, swap] & \Omega
\end{tikzcd}\]
Well, this square is a pullback square. Of course, that is not all we have to say. \NotaInterna{To be continued.}
\end{example}

%Let us introduce a small generalization of pullbacks and pushouts. Let \(\cat I\) be a category which has one object \(b\) and, for \(\lambda \in \Gamma\), one object \(x_\lambda\) one morphism \(f_\lambda : x_\lambda \to b\); there is no other morphisms than these ones and the identities in \(\cat I\). What are limits of functors \(F : \cat I \to \cat C\) in more explicit terms? Any limit of \(F\) consists of one object \(p\), one morphism \(h : p \to b\) and, for every \(\lambda \in \Gamma\), one morphism \(g_\lambda : p \to x_\lambda\) such that
%\[f_\lambda g_\lambda = h \text{ for every } \lambda \in \Gamma .\]
%We shall call this kind of limits {\em generalized pullbacks}. \NotaInterna{\q{Unofficial} name.} Generalized because the usual definition of pullback is obtained by choosing \(\Gamma\) to be a set of solely two elements. Observe also, as in the case of pullbacks, generalized pullbacks are products in certain comma categories. \NotaInterna{Talk about generalized pushouts too.}
%
%\begin{example}[Generalized pullbacks in \(\Set\)]
%Consider a family of sets \(\set{X_\lambda \mid \lambda \in \Gamma}\), one set \(T\) and functions \(f_\lambda : X_\lambda \to T\).
%\[E := \set{\left. x \in \prod_{\lambda \in \Gamma} X_\lambda \right\mid f_\alpha (x(\alpha)) = f_\beta (x(\beta)) \text{ for every } \alpha, \beta \in \Gamma} .\]
%We have then commutative squares
%\[\begin{tikzcd}
%E \ar["{p_\alpha}", r] \ar["{p_\beta}", d, swap] & X_\alpha \ar["{f_\alpha}", d] \\
%X_\beta \ar["{f_\beta}", r, swap] & T
%\end{tikzcd}\]
%where the \(p_\alpha\)-s are restrictions to \(E\) of the \(\proj_\alpha\)-s in Example~\ref{example:ProdOfSets}. Consider, for \(\alpha, \beta \in \Gamma\),
%\[\begin{tikzcd}
%Y \ar["{g_\alpha}", drr, bend left] \ar["{g_\beta}", ddr, bend right, swap] \\
%& E \ar["{p_\alpha}", r] \ar["{p_\beta}", d, swap] & X_\alpha \ar["{f_\alpha}", d] \\
%& X_\beta \ar["{f_\beta}", r, swap]                      & T
%\end{tikzcd}\]
%where \(f_\alpha g_\alpha = f_\beta g_\beta\). Thus \(g_\bullet (y) \in E\) for every \(y \in Y\), which fact motivates the following function
%\[Y \to E\,, \ y \to g_\bullet(y).\]
%After realizing how here we have generalized Example~\ref{example:PullbacksInSet}, continue and finish this example: \inlinethm{exercise}.
%\end{example}
%
%\begin{example}[Generalized pushouts in \(\Set\)]
%Recall Example~\ref{example:PushoutsInSet}, because we will need it. Consider a family of sets \(\set{X_\mu \mid \mu \in \Lambda}\), one set \(T\) and functions \(g_\mu : S \to X_\mu\), one for each \(\mu \in \Lambda\). As in the binary case, we have non commuting diagrams
%\[\begin{tikzcd}
%S \ar["{g_\delta}", r] \ar["{g_\eta}", d, swap] & X_\eta \ar["{\inj_\delta}", d] \\
%X_\eta \ar["{\inj_\eta}", r, swap]            & {\displaystyle\sum_{\mu \in \Lambda} X_\mu}
%\end{tikzcd}\]
%for \(\delta, \eta \in \Lambda\). This is just to push ourselves to the next move: consider the smallest relation \(\sim\) on the disjoint sum containing
%\[R := \set{(\inj_\delta g_\delta (x), \inj_\eta g_\eta(x)) \mid x \in S \text{ and } \delta, \eta \in \Lambda} .\]
%In this case, we have a commutative diagram
%\[\begin{tikzcd}
%S \ar["{g_\delta}", r] \ar["{g_\eta}", d, swap] & X_\delta \ar["{q_\delta}", d] \\
%X_\eta \ar["{q_\eta}", r, swap]                    & \frac{\sum_{\mu \in \Lambda} X_\mu}{\sim}
%\end{tikzcd}\]
%where, for \(\mu \in \Lambda\), we define \(q_\mu (x)\) to be the \(\sim\)-equivalence class of \(x \in X_\mu\). Take now one function \(h_\mu : X_k \to F\), for \(\mu \in \Lambda\), such that \(h_\delta g_\delta = h_\eta g_\theta\) for every \(\delta, \eta \in \Lambda\).
%\[\begin{tikzcd}
%S \ar["{g_\delta}", r] \ar["{g_\eta}", d, swap] & X_\delta \ar["{q_\delta}", d] \ar["{h_\delta}", ddr, bend left] \\
%X_\eta \ar["{q_\eta}", r, swap] \ar["{h_\eta}", drr, bend right, swap] & \frac{\sum_{\mu \in \Lambda} X_\mu}{\sim} \ar[dr, dashed] \\
%& & F
%\end{tikzcd}\]
%To construct the dotted function, we proceed similarly as in the binary case. We have the function
%\[\sum_{\mu \in \Lambda} h_\mu : \sum_{\mu \in \Lambda} X_\mu \to F\]
%which satisfies
%%\begin{align*}
%%& \left( \sum_{\mu \in \Lambda} h_\mu \right) (\inj_\delta g_\delta (x)) = h_\delta g_\delta (x) =  %\\
%%& = h_\eta g_\eta (x) = \left( \sum_{\mu \in \Lambda} h_\mu \right) (\inj_\eta g_\eta (x))
%%\end{align*}
%\[\left( \sum_{\mu \in \Lambda} h_\mu \right) (\inj_\delta g_\delta (x)) = h_\delta g_\delta (x) = h_\eta g_\eta (x) = \left( \sum_{\mu \in \Lambda} h_\mu \right) (\inj_\eta g_\eta (x))\]
%for \(x \in S\) and \(\delta, \eta \in \Lambda\). The function we are looking for is the one induce by \(\sum_{\mu \in \Lambda} h_\mu\). The continuation of this example is \inlinethm{exercise}.
%\end{example}

%\begin{exercise}
%Can you do something similar to what we have done with finite products? We are referring to Proposition~\ref{proposition:FiniteProdLeftCons}, Proposition~\ref{proposition:FiniteProdRightCons} and Corollary~\ref{corollary:ProdAssoc}.
%\end{exercise}

