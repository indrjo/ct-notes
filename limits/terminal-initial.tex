
\section{Terminal and initial objects}

\begin{definition}[Terminal \& initial objects]
  For \(\cat C\) category, the limits of the empty functor
  \(\nil \to \cat C\) are called {\em terminal objects} of \(\cat C\),
  whereas the colimits {\em initial objects}.
\end{definition}

Let us expand the definition above so that we can can look inside. A
cone over the empty functor \(\nil \to \cat C\) with vertex \(a\) is a
natural transformation
\[\naturaltr{}{\nil}{\cat C}{k_a}{}.\]
Here, the empty functor is \(k_a\) because there is at most one
functor \(\nil \to \cat C\). Again, because there must be a unique one,
our natural transformation is the empty transformation, viz the one
devoid of morphisms. A similar reasoning leads us to the following
explicit definition of terminal and initial object.

\begin{definition}[Terminal and initial objects, explicit]\label{definition:TerminalAndInitialExplicit}
  Let \(\cat C\) be a category.
  \begin{tcbitem}
  \item A {\em terminal object} of \(\cat C\) is an object \(1\) of
    \(\cat C\) such that for every object \(x\) of \(\cat C\) there
    exists one and only one \(x \to 1\) in \(\cat C\).
  \item An {\em initial object} of \(\cat C\) is an object \(0\) in
    \(\cat C\) such that for every object \(x\) in \(\cat C\) there
    exists one and only one morphism \(0 \to x\) in \(\cat C\).
  \end{tcbitem}
\end{definition}

% Examples time.

\begin{example}[Empty set and singletons]
  It may sound weird, but for every set \(X\) there does exist a
  function \(\nil \to X\); moreover, it is the unique one. To get this,
  think set-theoretically: a function is any subset of
  \(\nil \times X\) that has the property we know. But
  \(\nil \times X = \nil\), so its unique subset is \(\nil\). This set is a
  function from \(\nil\) to \(X\) since the statement
  \begin{quotation}
    for every \(a \in \nil\) there is one and only one \(b \in X\) such
    that \((a, b) \in \nil\)
  \end{quotation}
  is a {\em vacuous truth}.
  % \footnote{{\em Vacuous truths} are nice logical and mathematical
  % tricks. Suppose \(p\) is a predicate that is never true. Then
  % \q{\(p(x)\) for all \(x\)} is true.}
  So \(\nil\) is an initial object of \(\Set\). This case is quite
  particular, since the initial objects of \(\Set\) are actually equal
  to \(\nil\). \newline Now let us look for terminal objects in
  \(\Set\). Take an arbitrary set \(X\): there is exactly one function
  from \(X\) to any singleton, that is singletons are terminal object
  of \(\Set\). Conversely, by
  Proposition~\ref{proposition:LimitsAreIsomorphic}, the terminal
  objects of \(\Set\) must be singletons.
\end{example}

\begin{exercise}
  Trivial groups --- there is a unique way a singleton can be a group ---
  are either terminal and initial objects of \(\Grp\).
\end{exercise}

\begin{example}[The ring \(\integers\)]
  For if \(R\) is a ring, we write \(1_R\) to write the multiplicative
  identity of \(R\). A simple ring homomorphism is
  \[\phi : \integers \to R \,,\ \phi(r) := r 1_R .\]
  Assume now, \(\psi : \integers \to R\) is another ring homomorphism: then
  for every \(r \in \integers\) we have
  \[\psi(r) = r \psi\left(1_R\right) = r \phi\left(1_R\right) = \phi(r) .\]
  Thus we can conclude \(\integers\) is initial in \(\Ring\).
\end{example}

\begin{example}[Recursion]\label{example:Recursion}
  In Set Theory, there is a nice theorem, the {\em Recursion Theorem}:
  \begin{quotation}
    Let \((\naturals, 0, s)\) be a Peano Model, where
    \(0 \in \naturals\) and \(s : \naturals \to \naturals\) is its
    successor function. For every pointed set \(X\), \(a \in X\) and
    \(f : X \to X\) there exists one and only one function
    \(x : \naturals \to X\) such that \(x_0 = a\) and
    \(x_{s(n)} = f(x_n)\) for every \(n \in \naturals\).
  \end{quotation}
  Here, by Peano Model we mean a set \(\naturals\) that has one
  element, we write \(0\), stood out and a function
  \(s : \naturals \to \naturals\) such that, all this complying some
  rules:
  \begin{tcbenum}
  \item \(s\) is injective;
  \item \(s(x) \ne 0\) for every \(x \in \naturals\);
  \item for if \(A \subseteq \naturals\) has \(0\) and \(s(n) \in A\) for every
    \(n \in A\), then \(A = \naturals\).
  \end{tcbenum}
  We show now how we can involve Category Theory in this case. First
  of all, we need a category where to work.\newline The statement is about
  things made as follows:
  \begin{quotation}
    a set \(X\), one distinguished \(x \in X\) and one function
    \(f : X \to X\).
  \end{quotation}
  \NotaInterna{Is there a name for these things?}  We may refer to
  such new things by barely a triple \((X, a, f)\), but we prefer
  something like this:
  \[1 \functo{x} X \functo{f} X ,\] where \(1\) is any singleton, as
  usual. Peano Models are such things, with some additional
  properties. It is told about the existence and the uniqueness of a
  certain function. We do not want mere functions, of course: given
  \[1 \functo x X \functo f X \text{ and } 1 \functo y Y \functo g Y
    ,\] we take the functions \(r : X \to Y\) such that
  \[\begin{tikzcd}[row sep=tiny]
      & X \ar["f", r] \ar["r", dd] & X \ar["r", dd] \\
      1 \ar["x", ur] \ar["y", dr, swap] \\
      & Y \ar["g", r, swap] & Y
    \end{tikzcd}\] commutes and nothing else. \NotaInterna{Is there a
    name for such functions?} These ones are the things we want to be
  morphisms. Suppose given
  \[\begin{tikzcd}
      & X \ar["f", r] \ar["p", d] & X \ar["p", d] \\
      1 \ar["x", ur] \ar["y", r] \ar["z", dr, swap] & Y \ar["g", r] \ar["q", d] & Y \ar["q", d] \\
      & Z \ar["h", r, swap] & Z
    \end{tikzcd}\] where all the squares and triangles commute: thus
  we obtain the commuting
  \[\begin{tikzcd}[row sep=tiny]
      & X \ar["f", r] \ar["qp", dd] & X \ar["qp", dd] \\
      1 \ar["x", ur] \ar["z", dr, swap] \\
      & Z \ar["h", r, swap] & Z
    \end{tikzcd}\] This means that composing two morphisms as
  functions in \(\Set\) produces a morphism. This is how we want
  composition to defined in this context. This choice makes the
  categorial axioms automatically respected. We call this category
  \(\mathbf{Peano}\). \NotaInterna{Unless there is a better naming, of
    course.}\newline Being the environment set now, the Recursion Theorem
  becomes more concise:
  \begin{quotation}
    Peano Models are initial objects of \(\mathbf{Peano}\).
  \end{quotation}
  By Proposition~\ref{proposition:LimitsAreIsomorphic}, any other
  initial object of \(\mathbf{Peano}\) are isomorphic to some Peano
  Model: does this mean its initial objects are Peano Models?
  (Exercise.)
\end{example}

\begin{exercise}[Induction \(\lrarr\) Recursion]
  In \(\Set\), suppose you have
  \(1 \functo 0 \naturals \functo s \naturals\), where \(s\) is
  injective and \(s(n) \ne 0\) for every \(n \in \naturals\). Demonstrate
  that the following statements are equivalent:
  \begin{tcbenum}
  \item for if \(A \subseteq \naturals\) has \(0\) and \(s(n) \in A\) for every
    \(n \in A\), then \(A = \naturals\);
  \item \(1 \functo 0 \naturals \functo s \naturals\) is an initial
    object of \(\mathbf{Peano}\).
  \end{tcbenum}
  (1) \(\tto\) (2) proves the Recursion Theorem, whereas (2) \(\tto\)
  (1) requires you to codify a proof by induction into a
  recursion. Try it, it could be nice. \NotaInterna{Prepare
    hints\dots{}}
\end{exercise}

Limits (colimits) are terminal (initial) objects of appropriate
categories. Definition~\ref{definition:TerminalAndInitialExplicit}
does not make reference to limits and colimits as stated in
Definition~\ref{definition:LimitsAndColimits}: thus, you can say what
terminal and initial objects are and then tell what limits and
colimits are in terms of terminal and initial objects.

\begin{construction}[Category of cones]
  For \(\cat C\) category, let \(F : \cat I \to \cat C\) be a
  functor. Then we define the {\em category of cones} over \(F\) as
  follows.
  \begin{tcbitem}
  \item The objects are the cones over \(F\).
  \item For
    \(\alpha := \set{a \functo{\alpha_i} F(i)}_{i \in \obj{\cat I}}\) and
    \(\beta := \set{b \functo{\beta_i} F(i)}_{i \in \obj{\cat I}}\) two cones,
    the morphisms from \(\alpha\) to \(\beta\) are the morphisms
    \(f : a \to b\) of \(\cat C\) such that
    \[\begin{tikzcd}[row sep=tiny]
        a \ar["{\alpha_i}", dr] \ar["f", dd, swap] \\
        & F(i) \\
        b \ar["{\beta_i}", ur, swap]
      \end{tikzcd}\] commutes for every \(i \in \obj{\cat I}\).
  \item The composition of morphisms here is the same as that of
    \(\cat C\).
  \end{tcbitem}
  We write such category as \(\cn_F\). We define also the {\em
    category of cocones} over \(F\), written as \(\cocn_F\).
  \begin{tcbitem}
  \item The objects are the cocones over \(F\).
  \item For
    \(\alpha := \set{F(i) \functo{\alpha_i} a}_{i \in \obj{\cat I}}\) and
    \(\beta := \set{F(i) \functo{\beta_i} b}_{i \in \obj{\cat I}}\) cocones, the
    morphisms from \(\alpha\) to \(\beta\) are the morphisms
    \(f : a \to b\) of \(\cat C\) such that
    \[\begin{tikzcd}[row sep=tiny]
        & a \ar["{\alpha_i}", dl, swap] \ar["f", dd] \\
        F(i) \\
        & b \ar["{\beta_i}", ul]
      \end{tikzcd}\] commutes for every \(i \in \obj{\cat I}\).
  \item The composition of morphisms here is the same as that of
    \(\cat C\).
  \end{tcbitem}
  It is quite immediate in either of the cases to show that categorial
  axioms are verified.
\end{construction}

\begin{proposition}
  For \(\cat C\) category and \(F : \cat I \to \cat C\) functor,
  \begin{tcbitem}
  \item limits of \(F\) are terminal objects of \(\cn_F\) and
    viceversa.
  \item colimits of \(F\) are initial objects of \(\cocn_F\) and
    viceversa.
  \end{tcbitem}
\end{proposition}

\begin{proof}
  This is \inlinethm{exercise}.
\end{proof}

%%% Local Variables:
%%% mode: LaTeX
%%% TeX-master: "../CT"
%%% TeX-engine: luatex
%%% End:
