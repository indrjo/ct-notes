% !TEX root = ../adjointness.tex
% !TEX spellcheck = en_GB
% !TEX root = ../adjointness.tex

\section{Adjunctions and limits}

Let \(\cat I\) and \(\cat C\) be two categories. For every \(v \in \obj{\cat C}\) we have the {\em constant functor}
\[k_v : \cat I \to \cat C\]
where \(k_v(i) := v\) for every \(i \in \obj{\cat I}\) and \(k_v(f) := \id_v\) for every morphism \(f\) of \(\cat I\). Recall that \(\lambda : k_v \tto F\) being a limit of a functor \(F : \cat I \to \cat C\) means:
\begin{quotation}
for every \(\mu : k_v \tto F\) there exists one and only one \(f : a \to v\) of \(\cat C\) such that \(\mu_i = \lambda_i f\)
%\[\begin{tikzcd}[row sep=tiny]
%a \ar["{\mu_i}", dr] \ar["f", dd, swap] \\
%& F(i) \\
%v \ar["{\lambda_i}", ur, swap]
%\end{tikzcd}\]
commutes for every object \(i\) of \(\cat I\).
\end{quotation}
That is, if you put it in other words, it sounds like:
\begin{quotation}
there is a bijection
\[\cat C(a, v) \to [\cat I, \cat C](k_a, F)\]
taking \(f : a \to v\) to the natural transformation
\[\lambda_\bullet f : k_a \tto F\,, \ \lambda_\bullet f (i):= \lambda_i f .\]
\end{quotation}

There is a smell of adjunction situation here. Let us start with finding an appropriate pair of functors
\[\begin{tikzcd}[column sep=small] \cat C \ar[r, shift left] & {[\cat I, \cat C]} \ar[l, shift left] \end{tikzcd} .\]

One functor is already suggested:
\[\Delta : \cat C \to [\cat I, \cat C]\]
takes \(x \in \obj{\cat C}\) to the functor \(\cat I \to \cat C\) that maps every object to \(x\) and every morphism to \(\id_x\); then for \(i \in \obj{\cat I}\) define
\[\Delta \left(x \functo f y\right)\]
to be the natural transformation \(\Delta(x) \tto \Delta(y)\) amounting uniquely of \(f\).

From now on, assume \(\cat I\) is small and every functor \(\cat I \to \cat C\) has a limit. Now, in spite of not being strictly unique \NotaInterna{\q{strictly unique}\dots{} huh?}, all the limits of a given functor are isomorphic, so are the vertices: let us indicate by \(\lim F\) the vertex of any of the limits of \(F\). Now, take a natural transformation
\[\naturaltr{\xi}{\cat I}{\cat C}{F}{G} ;\]
%you have one morphism
%\[\lim F \to \lim G\]
%of \(\cat C\) induced by \(\eta\).
\(\lim F\) is the vertex of some limit
\[\set{\left. \lim F \functo{\lambda_i} F(i) \right\mid i \in \obj{\cat I}}\]
and \(\lim G\) is the vertex of a certain limit
\[\set{\left. \lim G \functo{\mu_i} G(i) \right\mid i \in \obj{\cat I}} .\] If we display all the stuff we have gathered so far, we have for \(i \in \obj{\cat I}\)
%\begin{equation}\begin{tikzcd}
%\lim F \ar["{\lambda_i}", r] & F(i) \ar["{\xi_i}", d] \\
%\lim G \ar["{\mu_i}", r, swap] & G(i)
%\end{tikzcd} \label{diagram:LimitOfNaturalTransf}\end{equation}
\begin{equation}\begin{tikzcd}
F(i) \ar["{\xi_i}", r] & G(i) \\
\lim F \ar["{\lambda_i}", u] & \lim G \ar["{\mu_i}", u, swap]
\end{tikzcd} \label{diagram:LimitOfNaturalTransf}\end{equation}
The universal property of limits ensures that there is one and only one morphism \(\lim F \to \lim G\) making the above diagram a commuting square. Let us call this morphism \(\lim \xi\). We have a functor
\[\lim : [\cat I, \cat C] \to \cat C\]
indeed. If you take \(F = G\) and \(\eta\) the identity of the functor \(F\) in~\eqref{diagram:LimitOfNaturalTransf}, then
\[\lim \id_F = \id_{\lim F} ,\]
obtained by uniquely employing the universal property of limit. Now take three functors \(F, G, H :  \cat I \to \cat C\) and two natural transformations \(F \functto \alpha G \functto \beta H\). To these functors are associated the respective limits
\[\begin{aligned}
& \set{\left. \lim F \functo{\lambda_i} F(i) \right\mid i \in \obj{\cat I}} \\
& \set{\left. \lim G \functo{\mu_i} G(i) \right\mid i \in \obj{\cat I}} \\
& \set{\left. \lim H \functo{\eta_i} H(i) \right\mid i \in \obj{\cat I}}
\end{aligned}\]
so that we have commuting squares glued together:
%\[\begin{tikzcd}
%\lim F \ar["{\lambda_i}", r] \ar["{\lim \alpha}", d, swap] & F(i) %\ar["{\alpha_i}", d] \\
%\lim G \ar["{\mu_i}", r] \ar["{\lim \beta}", d, swap] & G(i) \ar["{\beta_i}", d] \\
%\lim H \ar["{\eta_i}", r] & H(i)
%\end{tikzcd}\]
\[\begin{tikzcd}
F(i) \ar["{\alpha_i}", r] & G(i) \ar["{\beta_i}", r] & H(i) \\
\lim F \ar["{\lambda_i}", u] \ar["{\lim \alpha}", r, swap] & \lim G \ar["{\mu_i}", u] \ar["{\lim \beta}", r, swap] & \lim H \ar["{\eta_i}", u, swap]
\end{tikzcd}\]
We have for every \(i \in \obj{\cat I}\)
\[\eta_i \lim \beta \lim \alpha = \beta_i \mu_i \lim \alpha = \beta_i \alpha_i \lambda_i ;\]
then, by how it is defined the limit of a natural transformation, it must be
\[\lim(\beta\alpha) = \lim\beta \lim \alpha .\]

The following proposition pushes all this discourse to a conclusion.

\begin{proposition}
There is an adjunction
\[\begin{tikzcd}
\phantom{[\cat I, \cat D]}\lapbox{-6pt}{\(\cat C\)}
\ar["\Delta"{name=delta}, r, bend left=15pt] & {[\cat I, \cat D]} \ar["\lim"{name=lim}, l, bend left=15pt] \ar["\perp"{description}, phantom, from=delta, to=lim]
\end{tikzcd}\]
\end{proposition}

\begin{proof}
\NotaInterna{Yet to be \TeX{}-ed\dots{}}
\end{proof}
