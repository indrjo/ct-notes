% !TEX program = lualatex
% !TEX spellcheck = en_GB
% !TEX root = ../adjointness.tex

\section{Other exercises}

\begin{remark}[About the next exercise]
Let \(\cat C\) be a category with binary products and consider one object \(c\) of \(\cat C\). For \(a\) object in \(\cat C\) choose a product of \(a\) and \(c\)
\[\begin{tikzcd}[row sep=tiny]
& a \times c \ar["{p_a}", dl, swap] \ar["{q_a}", dr] & \\
a & & c
\end{tikzcd}\]
in \(\cat C\). In that context, there is a sensible way to define the functor
\[(\times c) : \cat C \to \cat C .\]
(If you have not done the exercise of the chapter of limits that about this construction, do it now.)
\end{remark}

\begin{exercise}[Exponential objects]
In the category \(\cat C\) that has binary products, the {\em exponential object} of two objects \(a\) and \(b\) of  \(\cat C\) is any
\begin{tcbitem}
\item object of \(\cat C\), that we choose label as \(b^a\)
\item a morphism \(\ev : b^a \times a \to b\), that we call {\em evaluation map}
\end{tcbitem}
such that \(\ev\) is a terminal object of \((\times a) \downarrow b\). A category \(\cat C\) is said to \q{have exponentials} whenever for every \(a, b \in \obj{\cat C}\) there is in \(\cat C\) the corresponding exponential object.

\begin{tcbenum}
\item Find an adjunction
\[\begin{tikzcd}
\cat C \ar["{(\times c)}"{name=L}, r, bend left] & \cat C \ar["{\square^c}"{name=R}, l, bend left]
\ar["\perp"{description}, phantom, from=L, to=R]
\end{tikzcd}\]
This boils down to introducing an appropriate functor \(\square^c\) using uniquely the definition of exponential object.
\item Assume \(\cat C\) has also an initial object \(0\) and terminal object \(1\). Prove the following statements.
\begin{enumerate}[leftmargin=*,label=(\roman*)]
\item \(a \times 0 \cong 0 \times a \cong a\) for every object \(a\) of \(\cat C\).
\item For every \(a \in \obj{\cat C}\), if there is some morphism \(a \to 0\), then \(a \cong 0\).
\item Any morphism \(0 \to a\) is monic for \(a \in \obj{\cat C}\).
\item If \(0 \cong 1\), then all the objects of \(\cat C\) are isomorphic.
\item \(a^1 \cong a\), \(a^0 \cong 0\) and \(1^a \cong 1\) for every \(a \in \obj{\cat C}\).
\end{enumerate}
\end{tcbenum}

\end{exercise}
