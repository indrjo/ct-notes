
\section{Exponentiation}

Let \(\cat C\) be a category with binary products. Consider one object
\(c\) of \(\cat C\). For \(a\) object in \(\cat C\) choose
\[\begin{tikzcd}[row sep=tiny]
    & a \times c \ar["{p_a}", dl, swap] \ar["{q_a}", dr] & \\
    a & & c
  \end{tikzcd}\] to be any of the products of \(a\) and \(c\). That
being said, we will work now to construe a functor
\[(\times c) : \cat C \to \cat C .\] As the notation hints, we make the
convention
\[(\times c) (x) := x \times c\] for \(x\) object of \(\cat C\). \NotaInterna{In
  the zeroth chapter, remember to introduce this kind of notation.}
Now, we shall involve morphisms too. \NotaInterna{In the section of
  products, remember to talk about the the product of morphisms.} If
we take any \(f : a \to b\) of \(\cat C\), then we instruct
\((\times c)\) on morphisms as follows:
\[(\times c) (f) := f \times \id_c .\] \NotaInterna{In the section of products,
  remember to talk about the the product of morphisms.} Functoriality,
in this case, directly descends from what we have said in the section
about products. \NotaInterna{Remember to \TeX{} that part too.}
% We dedicate some lines to prove \((\times c)\) is actually a functor.
% \NotaInterna{Working about this\dots{}}

\begin{definition}[Exponential object]
  In the category \(\cat C\) that has binary products, the {\em
    exponential object} of two objects \(a\) and \(b\) of \(\cat C\)
  is any
  \begin{tcbitem}
  \item object of \(\cat C\), we write as \(b^a\)
  \item a morphism \(\ev : b^a \times a \to b\), the {\em evaluation}
  \end{tcbitem}
  such that \(\ev\) is a terminal object of
  \((\times a) \downarrow b\). A category \(\cat C\) is said to \q{have
    exponentials} whenever for every \(a, b \in \obj{\cat C}\) there is
  in \(\cat C\) the corresponding exponential object.
\end{definition}

We can involve adjunctions in this discourse now! In fact, the
definition gives a bijection
\[\cat C(a \times c, b) \cong \cat C\left(a, b^c\right)\]
for \(a, b, c \in \obj{\cat C}\). Thus, first of all, we shall try to
arrive to a situation like this
\[\begin{tikzcd}
    \cat C \ar["{(\times c)}", r, shift left] & \cat C \ar["?", l, shift
    left]
  \end{tikzcd}\] so that we can see if
\((\times c) \dashv ?\).\newline The functor labelled with a question mark comes from a
close analysis of the the new objects just introduced. Consider a
morphism \(f : a \to b\) and a third object \(c\) of \(\cat C\). Assume
also, \(\cat C\) has exponentials. If we write \(\ev_a\) and \(\ev_b\)
for the evaluations associated to \(a^c\) and \(b^c\), we we have
\[\begin{tikzcd}
    a^c \times c \ar["{\ev_a}", d, swap] & b^c \times c \ar["{\ev_b}", d] \\
    a \ar["f", r, swap] & b
  \end{tikzcd} .\] Composing \(\ev_a\) and \(f\) and the definition of
exponential objects yield a unique morphism \(a^c \to b^c\) of
\(\cat C\), we denote \(f^c\), making
\[\begin{tikzcd}
    a^c \times c \ar["{\ev_a}", d, swap] \ar["{f^c \times \id_c}", r] & b^c \times c \ar["{\ev_b}", d] \\
    a \ar["f", r, swap] & b
  \end{tikzcd}\] commute. Indeed, we have the functor
\(\square^c : \cat C \to \cat C\) with
\[\square^c (x) := x^c\]
for \(x\) object of \(\cat C\) and
\[\square^c (f) := f^c\]
for \(f : a \to b\) in \(\cat C\).

\begin{exercise}
  Under the hypothesis that \(\cat C\) has exponentials, you can
  provide functors \(c^\bullet : \opcat C \to \cat C\), using exponential
  objects. This exercise is not essential for this section.
\end{exercise}

The conclusion is the following theorem.

\begin{proposition}
  \((\times c) \dashv \square^c\) for every \(c \in \obj{\cat C}\).
\end{proposition}

\begin{proof}
  There is not much work left to do: we have two functors running in
  opposite directions and we have bijections
  \(\cat C(a \times c, b) \to \cat C\left(a, b^c\right)\), one for every
  \(c \in \obj{\cat C}\); we have just to verify the naturality
  condition.
\end{proof}

\begin{proposition}
  A category \(\cat C\) with binary products has exponentials if and
  only if for every \(c \in \obj{\cat C}\) the functor
  \((\times c) : \cat C \to \cat C\) has a right adjoint. \NotaInterna{More
    details\dots{}}
\end{proposition}

\begin{proof}
  At this point, one of the implications is already demonstrated. The
  remaining can be readily derived using the constructions of the
  section about units and counits. \NotaInterna{That section is a
    mess\dots{}} Indeed, the evaluation morphisms
  \[\ev : a^c \times c \to a\]
  are terminal objects of
  \([(\times c) \circ \square^c] \downarrow a\) and form a natural transformation
  \[(\times c) \circ \square^c \tto \id_{\cat C} .\] \NotaInterna{More
    details\dots{}} \NotaInterna{Make additions to the section about
    units and co-units.}
\end{proof}

\NotaInterna{And now\dots{} Cartesian closed categories?}

%%% Local Variables:
%%% mode: LaTeX
%%% TeX-master: "../CT"
%%% TeX-engine: luatex
%%% End:
