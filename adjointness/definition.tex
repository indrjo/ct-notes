% !TEX program = lualatex
% !TEX spellcheck = en_GB
% !TEX root = ../adjointness.tex

\section{Isolating the concept}

\begin{example}[Defining linear functions, part I]\label{example:DefineLinearFunctionsI}
Consider two vector spaces \(V\) and \(W\) (over the same field) and the problem:
%
\begin{quotation}
how can we define a linear function \(f : V \to W\)?
\end{quotation}
%
There is a well known theorem that says that prescribing the images of the elements of the base \(S\) determines uniquely a linear function \(V \to W\). More precisely, the theorem sounds like this:
%
\begin{quotation}
Let \(V\) and \(W\) two vector spaces, both over a field \(k\), and let \(S\) be a base of \(V\). Let us write \(i\) for the inclusion \(S \hookrightarrow V\). Then for every function \(\phi : S \to W\) there exists one and only one linear function \(f : V \to W\) such that
\begin{equation}\begin{tikzcd}
S \ar["{i}", r, hook] \ar["\phi", dr, swap] & V \ar["f", d] \\
& W
\end{tikzcd}\label{diagram:FunctionsExtendsToALinearMapI}\end{equation}
commutes.
\end{quotation}
%
The statement is equivalent to saying that the function
\begin{equation}
\Vect_k(V, W) \to \Set(S, W) \,,\ f \to fi \label{function:AdjBijSetVectI}
\end{equation}
is a bijection.\newline
Let us reason about the theorem. Fist of all, it is about a {\em function} \(\phi : S \to W\), pointing to a vector space \(W\): the morphisms of \(\Set\) do not care whether the sets have an additional structure. Let us say that \(\phi\) is a {\em function} from \(S\) to \(W\) \q{downgraded} from the status of vector space to the one of set. On the other hand, from a set we construct an actual vector space, this is what being a base means.
%
\begin{figure}
\centering
\input{adjointness/free-forgetful-adjunction.qtikz}
\caption{\(F\) upgrades sets to vector spaces; \(U\) does the opposite, that is downgrades}
\end{figure}
%
Indeed, behind the scenes two functors
\[\begin{tikzcd}
\Set \ar["F", r, shift left] & \Vect_k \ar["U", l, shift left]
\end{tikzcd}\]
are moving:
%
\begin{tcbitem}
\item \(F\) the functor that {\em constructs} a vector space \(F(S)\) from set \(S\) it and extends a function of bases \(\alpha : S \to T\) to a genuine linear functions \(F(\alpha) : F(S) \to F(T)\), as we have already explained.
\item \(U\) is the functor that {\em forgets}: that is takes a vector space and returns the set of vectors (thus no addition, no external product and no vector space axioms); it takes a linear function and returns the same function, but observe the homomorphism property cannot make sense any longer in \(\Set\).
\end{tcbitem}
%
Let us rewrite the quoted theorem to make it more aware of these functors. First, let us rewrite \(i : S \to V\) as \(i_S : S \to U(F(S))\) and \(\phi : S \to W\) as \(\phi : S \to U(W)\): we have just wrapped \(V\) and \(W\) with \(U\). So far, the situation ca be depicted as
%
\[\begin{tikzcd}
S \ar["{i_S}", r, hook] \ar["\phi", dr, swap] & U(F(S)) \\
& U(W)
\end{tikzcd}\]
%
We are not loosing the commutative diagram~\eqref{diagram:FunctionsExtendsToALinearMapI}! The composition of a \q{function} \(S \to V\) with a \q{linear function} \(V \to W\) which gives as result a \q{function} \(S \to W\): in this composition we not care about linearity anymore, thus let us wrap the \(f : V \to W\) with \(U\), so that it can fit in a commutative diagram:
%
\begin{quotation}
Let \(V\) and \(W\) two vector spaces over a field \(k\), and let \(S\) be a base of \(V\). In this context, \(V = F(S)\). Moreover, write \(i_S\) for the inclusion function \(S \hookrightarrow U(F(S))\). Then for every function \(\phi : S \to U(W)\) there exists one and only one linear function \(f : V \to W\) for which
\begin{equation}\begin{tikzcd}
S \ar["{i_S}", r, hook] \ar["\phi", dr, swap] & U(F(S)) \ar["U(f)", d] \\
& U(W)
\end{tikzcd}\label{diagram:FunctionsExtendsToALinearMapII}\end{equation}
is a commutative diagram of \(\Set\).
\end{quotation}
%
The restyling touches also the bijections in~\eqref{function:AdjBijSetVectI}:
\begin{equation}\xi_{S,W} : \Vect_k(F(S), W) \to \Set(S, U(W)) \,,\ \xi_{S,W}(f) := U(f)i_S \label{function:AdjBijSetVectII}\end{equation}
We need to pause the example a bit now and resume it later.
\end{example}

\begin{construction}
Let \(\cat C\) and \(\cat D\) be two locally small categories and two functors
\[\begin{tikzcd} \cat C \ar["L", r, shift left] & \cat D \ar["R", l, shift left] \end{tikzcd}\]
We have then the functor
\[\cat C(\hole, R(\hole)) : \opcat C \times \cat D \to \Set\]
that maps objects \((x, y)\) to \(\cat C(x, R(y))\) and pairs of morphisms
\[\left(\begin{tikzcd}[row sep=small](x, y) \ar["{(f, g)}", d, swap] \\ (x', y')\end{tikzcd}\right) = \left(\begin{tikzcd}[row sep=small]x \\ x' \ar["f", u]\end{tikzcd}, \begin{tikzcd}[row sep=small]y \ar["g", d] \\ y'\end{tikzcd}\right)\]
to functions
\[\begin{aligned}
\cat C(x, R(y)) &\to \cat C(x', R(y')) \\
h &\to R(g) h f
\end{aligned}\]
We have also the functor
\[\cat D(L(\hole), \hole) : \opcat C \times \cat D \to \Set\]
that maps \((x, y)\) to \(\cat C(L(x), y)\) and pairs of morphisms
\[\left(\begin{tikzcd}[row sep=small](x, y) \ar["{(f, g)}", d, swap] \\ (x', y')\end{tikzcd}\right) = \left(\begin{tikzcd}[row sep=small]x \\ x' \ar["f", u]\end{tikzcd}, \begin{tikzcd}[row sep=small]y \ar["g", d] \\ y'\end{tikzcd}\right)\]
to functions
\[\begin{aligned}
\cat D(L(x), y) &\to \cat D(L(x'), y') \\
h &\to g h L(f) .
\end{aligned}\]
\end{construction}

\begin{example}[Defining linear functions, part II]\label{example:DefineLinearFunctionsII}
We return to the last example. The functions \(\xi_{S,W}\) form a natural isomorphism
\[\xi : \Vect_k(F(\hole), \hole) \tto \Set(\hole, U(\hole))\]
We know the components are bijections, hence it remains to check it is a natural transformation. Thus consider the diagram
%
\[\begin{tikzcd}
\Vect_k(F(S), W) \ar["{\Vect_k(F(\alpha), f)}", d, swap] \ar["{\xi_{S,W}}", r] & \Set(S, U(W)) \ar["{\Set(\alpha, U(f))}", d] \\
\Vect_k(F(S'), W') \ar["{\xi_{S',W'}}", r, swap] & \Set(S', U(W'))
\end{tikzcd}\]
%
and we show it is commutative. A linear function \(h : F(S) \to W\) goes to the function \(U(h)i_S : S \to U(W)\), which is sent to the function \(U(f) (U(h)i_S) \alpha : S' \to U(W')\). By functoriality, \(U(f) (U(h)i_S) \alpha = U(fh) i_S \alpha\). On the other way, \(h\) goes to the linear function \(f h F(\alpha) : F(S') \to W'\) which goes to \(U(f h F(\alpha)) i_{S'} : S' \to U(W')\). Here, \(U(f h F(\alpha)) i_{S'} = U(fh) U(F(\alpha))i_{S'}\). Thus, to verify that the functions \(S' \to U(W')\) here are equal, one could just verify that
\[\begin{tikzcd}
S \ar["{i_S}", r] & U(F(S)) \\
S' \ar["\alpha", u] \ar["{i_{S'}}", r, swap] & U(F(S')) \ar["U(F(\alpha))", u, swap]
\end{tikzcd}\]
commutes, which is immediate.
\end{example}

Example~\ref{example:DefineLinearFunctionsI} and~\ref{example:DefineLinearFunctionsII} of the introduction should have triggered your attention. If not, look at them closely now: after the initial restyling of one result of Linear Algebra, it is just a matter of categories and functors. 

\begin{construction}
Let \(\cat C\) and \(\cat J\) two categories, \(a\) one of its objects and take a functor \(F : \cat J \to \cat C\). We have the category \(a {\downarrow} F\) made as follows:
\begin{tcbitem}
\item the objects are the morphisms \(a \to F(x)\) of \(\cat C\), with \(x\) being an object of \(\cat J\);
\item the morphisms from \(f : a \to F(x)\) to \(g : a \to F(y)\) are the morphisms \(h : x \to y\) of \(\cat J\) such that
\[\begin{tikzcd}[row sep=small]
 & F(x) \ar["{F(h)}", dd] \\
a \ar["f", ur] \ar["g", dr, swap] \\
 & F(y)
\end{tikzcd}\]
commutes;
\item the composition is that of \(\cat J\). 
\end{tcbitem}
\end{construction}

\begin{proposition}\label{proposition:AdjunctionAsInitialInCommaCategory}
Suppose given two locally small categories \(\cat C\) and \(\cat D\), two functors
\[\begin{tikzcd}[column sep=small]
\cat C \ar["L", r, shift left] & \cat D \ar["R", l, shift left]
\end{tikzcd}\]
and a natural transformation \(\eta : \id_{\cat C} \tto RL\) such that \(\eta_x : x \to RL(x)\) is initial in \(x {\downarrow} R\) \NotaInterna{did we introduce comma categories?} for every \(x \in \obj{\cat C}\).  Then, for \(x \in \obj{\cat C}\) and \(y \in \obj{\cat D}\), the functions
\[\alpha_{x,y} : \cat D(L(x), y) \to \cat C(x, R(y)) \,,\ \alpha_{x,y}(f) := R(f)\eta_x\]
form an adjunction \(\alpha : L \dashv R\).
\end{proposition}

\begin{exercise}
Look at Example~\ref{example:DefineLinearFunctionsI} and~\ref{example:DefineLinearFunctionsII}: isolate what in the proposition is the natural transformation \(\eta\). Can you prove the theorem by yourself?
\end{exercise}

\begin{proof}[Proof of Proposition~\ref{proposition:AdjunctionAsInitialInCommaCategory}]
The fact that \(\eta_x\) is initial object implies that these function are all bijective. Now, we just need to verify the transformation is natural. Take \(x, x' \in \obj{\cat C}\), \(y, y' \in \obj{\cat D}\), \(f \in \cat C(x', x)\) and \(g \in \cat D(y, y')\) and examine the square
\[\begin{tikzcd}[column sep=large]
\cat D(L(x), y) \ar["{\cat D(L(f), g)}", d, swap] \ar["{\alpha_{x,y}}", r] & \cat C(x, R(y)) \ar["{\cat C(f, R(g))}", d] \\
\cat C(x', R\left(y'\right)) \ar["{\alpha_{x',y'}}", r, swap] & \cat C(x', R(y'))
\end{tikzcd}\]
For \(h \in \cat D(L(x), y)\), we have
\begin{align*}
& \cat C(f, R(g)) (\alpha_{x,y} (h)) = R(g) R(h) \eta_x f = R(gh) \eta_x f \\
& \alpha_{x',y'} (\cat D (L(f), g)(h)) = R(guL(f)) \eta_{x'} = R(gu) RL(f) \eta_{x'}
\end{align*}
By the naturality of \(\eta\), we have \(\eta_x f = RL(f) \eta_{x'}\), and the proof ends here.
\end{proof}

%Here is another example in the spirit of the initial example about vector spaces and of the proposition just proved here.
%
%\begin{example}[Isomorphism Theorem for Set Theory]
%We have defined \(\Eqv\) earlier, recall it here. Let us introduce the functor
%\[E : \Set \to \Eqv\]
%that maps a set \(X\) to a setoid \((X, =_X)\), where \(=_X\) is the equality relation over \(X\), and a function \(f : X \to Y\) to itself regarded as morphisms of setoids. Besides, Corollary~\ref{cor:SetIso2} gives a functor
%\[P : \Eqv \to \Set .\]
%Consequently, Proposition~\ref{proposition:SetIso1} can be easily rephrased more concisely as:
%\begin{quotation}
%the canonical projection \((X, \sim) \to E(P(X, \sim))\) is initial in \((X, \sim) \downarrow E\).
%\end{quotation}
%\end{example}

There is the dual of Proposition~\ref{proposition:AdjunctionAsInitialInCommaCategory} as well.

\begin{proposition}\label{proposition:AdjunctionAsTerminalInCommaCategory}
Suppose given two locally small categories \(\cat C\) and \(\cat D\), two functors
\[\begin{tikzcd}[column sep=small]
\cat C \ar["L", r, shift left] & \cat D \ar["R", l, shift left]
\end{tikzcd}\]
and a natural transformation \(\theta : LR \tto \id_{\cat D}\) such that \(\theta_y : LR(y) \to y\) is terminal in \(L {\downarrow} y\) for every \(y \in \obj{\cat C}\).  Then, for \(x \in \obj{\cat C}\) and \(y \in \obj{\cat D}\), the functions
\begin{align*}
\cat C(x, R(y)) & \to \cat D(L(x), y) \\
f &\to \theta_y L(f)
\end{align*}
form an adjunction \(L \dashv R\).
\end{proposition}

\begin{exercise}
The proof is left to you.
\end{exercise}

%The natural isomorphism \(\xi\) of the is what we have been striving for along the whole example. The phenomenon has a name.
%
%\begin{definition}[Adjunctions]
%Consider two locally small categories and two functors as in
%\[\begin{tikzcd}[column sep=small]
%\cat C \ar["L", r, shift left] & \cat D \ar["R", l, shift left]
%\end{tikzcd}\]
%An {\em adjunction} from \(L\) to \(R\) any natural isomorphism
%%
%\begin{equation}
%\begin{tikzcd}[column sep=large]
%\op{\cat C} \times \cat D
%  \ar["{\cat D(L(\hole), \hole)}"{name=A}, r, bend left=35] 
%  \ar["{\cat C(\hole, R(\hole))}"{name=B}, r, bend right=35, swap]
%  & \Set
%\ar["\alpha", from=A, to=B, natural]
%\end{tikzcd}
%\label{diagram:AdjunctionI}\end{equation}
%We say \(L\) is the {\em left adjoint} and \(R\) is the {\em right adjoint}: the reason behind the naming comes from when we write the bijection
%\[\cat D (L(x), y) \cong \cat C (x, R(y))\]
%\(L\) occurs in \(\cat D (L(x), y)\) applied to the argument on the left, while \(R\) appears in \(\cat C (x, R(y))\) applied on the right.
%Adjunctions are usually written as \(\alpha : L \dashv R\). Sometimes \(L \dashv R\) is written to mean that there is an adjunction in between without specifying which one. If on the the paper we are writing there is space, we can write something like
%%
%\begin{equation}
%\begin{tikzcd}
%\cat C \ar["L"{name=L}, r, bend left] & \cat D \ar["R"{name=R}, l, bend left]
%\ar["\perp"{description}, from=L, to=R, phantom]
%\end{tikzcd}
%\label{diagram:AdjunctionII}\end{equation}
%%
%which has in addition shows the categories involved.
%\end{definition}
%
%%\begin{remark}
%%In~\eqref{diagram:AdjunctionII} there is not the information of direction, as there is in~\eqref{diagram:AdjunctionI}. An adjunction is a natural isomorphism, but we must be explicit when we are using the components of the adjunction and not make the reader waste to much time for inferring what we are doing. There is some abuse of notation that help.
%%
%%One is dropping the pedices used to distinguish the components of a natural transformation. \NotaInterna{Talk about that!} Although this requires more space, we sometimes embed type signatures, being that more transparent and comfortable for both writers and readers. Writing
%%%
%%\[\alpha\left(L(x) \functo f y\right)\]
%%%
%%is more immediate than \(\alpha_{x,y}(f)\) and \(\alpha(f)\).
%%
%%Another abuse is when we are (willingly) careless about directions. If we have the adjunction~\eqref{diagram:AdjunctionI}, then we may sometimes happen to write
%%\[\alpha (g) \quad\text{for } g : x \to R(y) .\]
%%Of course, we mean \(\inv\alpha_{x,y}(g)\), bear with us.
%%
%%However, observe how the abuse drives to something that may baffle:
%%\[\alpha(\alpha(h)) = h\]
%%We are not saying that \(\alpha\) is idempotent. If \(h : L(x) \to y\), then it is to be read like
%%\[\inv\alpha_{x, y} \left(\alpha_{x, y} (h)\right) = h .\] 
%%If \(h : x \to R(y)\), then it is to be read like
%%\[\alpha_{x, y} \left(\inv\alpha_{x, y} (h)\right) = h .\] 
%%\end{remark}
%
%We will soon return to the introductory example later; as for now, let us indulge a bit more on the definition of adjunction.
%
%\begin{exercise}[Partial functions]
%For \(A\) and \(B\) sets, a {\em partial function} from \(A\) to \(B\) is relation \(f \subseteq A \times B\) with the property
%\begin{center}
%for every \(x \in A\) and \(y_1, y_2 \in B\), if \((x, y_1) \in f\) and \((x, y_2) \in f\) then \(y_1 = y_2\).
%\end{center}
%We want to compose partial functions as well: provided \(f \in \Par(A, B)\) and \(g \in \Par(B, C)\),
%\[gf \coloneq \set{(x, y) \in A \times C \mid (x, z) \in f \text{ and } (z, y) \in g \text{ for some } z \in B} .\]
%It is immediate to verify \(\Par\) complies the rules that make it a category. Indeed, this is the {\em category of partial functions}, written as \(\Par\): here, the objects are sets and the morphisms are partial functions.\newline
%Suppose given a partial function \(f : A \to B\). For every \(x \in A\) the possibilities are two: there is one element of \(B\) bound to is, and we write it \(f(x)\), or none. What if we considered {\em no value} as an output value? Provided two sets \(A\) and \(B\) and a partial function \(f : A \to B\), we assign an actual function
%\[\bar f : A \to B+1 \,, \ \bar f(x) \coloneq \begin{cases} f(x) & \text{if \(x\) has an element of \(B\) bound} \\ \ast & \text{otherwise} \end{cases}\]
%where \(1 \coloneq \set{\ast}\) with \(\ast\) designating the absence of output. It is quite simple to show that
%\[\Par(A, B) \to \Set(A, B+1)\,,\ f \to \bar f\]
%is a bijection for every couple of sets \(A\) and \(B\). Now it's up to you to categorify this: find two functors that make an adjunction
%\[\begin{tikzcd}
%\Set \ar["I"{name=I}, r, bend left] & \Par \ar["J"{name=J}, l, bend left]
%\ar["\perp"{description}, phantom, from=I, to=J]
%\end{tikzcd}\]
%\end{exercise}
