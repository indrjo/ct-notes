
\section{Definition, units and counits}

\begin{definition}[Adjunctions]
  Consider two locally small categories and two functors as in
  \[\begin{tikzcd}[column sep=small]
      \cat C \ar["L", r, shift left] & \cat D \ar["R", l, shift left]
    \end{tikzcd}\] An {\em adjunction} from \(L\) to \(R\) any natural
  isomorphism
%
  \begin{equation}
    \begin{tikzcd}[column sep=large]
      \op{\cat C} \times \cat D \ar["{\cat D(L(\hole), \hole)}"{name=A}, r,
      bend left=35] \ar["{\cat C(\hole, R(\hole))}"{name=B}, r, bend
      right=35, swap] & \Set \ar["\alpha", from=A, to=B, natural]
    \end{tikzcd}
    \label{diagram:AdjunctionI}\end{equation}
  We say \(L\) is the {\em left adjoint} and \(R\) is the {\em right
    adjoint}: the reason behind the naming comes from when we write
  the bijection
  \[\cat D (L(x), y) \cong \cat C (x, R(y))\]
  \(L\) occurs in \(\cat D (L(x), y)\) applied to the argument on the
  left, while \(R\) appears in \(\cat C (x, R(y))\) applied on the
  right.  Adjunctions are usually written as \(\alpha : L \dashv R\). Sometimes
  \(L \dashv R\) is written to mean that there is an adjunction in between
  without specifying which one. If on the the paper we are writing
  there is space, we can write something like
%
  \begin{equation}
    \begin{tikzcd}
      \cat C \ar["L"{name=L}, r, bend left] & \cat D \ar["R"{name=R},
      l, bend left] \ar["\perp"{description}, from=L, to=R, phantom]
    \end{tikzcd}
    \label{diagram:AdjunctionII}\end{equation}
%
  which has in addition shows the categories involved.
\end{definition}

\begin{remark}
  In~\eqref{diagram:AdjunctionII} there is not the information of
  direction, as there is in~\eqref{diagram:AdjunctionI}. An adjunction
  is a natural isomorphism, but we must be explicit when we are using
  the components of the adjunction and not make the reader waste to
  much time for inferring what we are doing. There is some abuse of
  notation that help.

  One is dropping the pedices used to distinguish the components of a
  natural transformation. \NotaInterna{Talk about that!} Although this
  requires more space, we sometimes embed type signatures, being that
  more transparent and comfortable for both writers and
  readers. Writing
%
  \[\alpha\left(L(x) \functo f y\right)\]
%
  is more immediate than \(\alpha_{x,y}(f)\) and \(\alpha(f)\).

  Another abuse is when we are (willingly) careless about
  directions. If we have the adjunction~\eqref{diagram:AdjunctionI},
  then we may sometimes happen to write
  \[\alpha (g) \quad\text{for } g : x \to R(y) .\]
  Of course, we mean \(\inv\alpha_{x,y}(g)\), bear with us.

  However, observe how the abuse drives to something that may baffle:
  \[\alpha(\alpha(h)) = h\]
  We are not saying that \(\alpha\) is idempotent. If
  \(h : L(x) \to y\), then it is to be read like
  \[\inv\alpha_{x, y} \left(\alpha_{x, y} (h)\right) = h .\]
  If \(h : x \to R(y)\), then it is to be read like
  \[\alpha_{x, y} \left(\inv\alpha_{x, y} (h)\right) = h .\]
\end{remark}

We will soon return to the introductory example later; as for now, let
us indulge a bit more on the definition of adjunction.

\newcommand\curry{\operatorname{curry}}

\begin{example}[Prescribing functions via currying]\label{example:CurryingFunctions}
  We can build a bijection
  \begin{equation}\begin{aligned}
    & \curry : \Set(A \times B, C) \to \Set(A, \Set(B, C)) \\
    & \curry f := \lambda x. \lambda y. f(x, y)
  \end{aligned}\label{function:Curry}\end{equation}
The truth is it is more than a bijection, but in order to realise that
we need to do some reverse engineering: more precisely, the functions
\(\curry\) should form a natural transformation from two functors
\[\begin{tikzcd}[column sep=small] \Set \ar[r, shift left] & \Set
    \ar[l, shift left] \end{tikzcd}\] Which functors, though? Look
at~\eqref{function:Curry} for a hint. The functor pointing leftward
takes sets \(A\) to \(A \times B\), while the opposite sends sets
\(C\) to \(\Set(B, C)\). Indeed the functors
\[\begin{tikzcd}
    \Set \ar["{(\times B)}", r, shift left] & \Set \ar["{\Set(B, \hole)}",
    l, shift left]
  \end{tikzcd}\] fit in the adjunction situation
\[\curry : (\times B) \dashv \Set(B, \hole) .\]
(As you can see, the set \(B\) is fixed while \(A\) and \(C\) while
varying provides the components of the natural isomorphism.) Let us
explicitly check naturality of \(\curry\): consider
\[\begin{tikzcd}
    \Set(A \times B, C) \ar["{\Set(f \times \id_B, g)}", d, swap] \ar["\curry", r] & \Set(A, \Set(B, C)) \ar["{\Set(f, \Set(B, g))}", d] \\
    \Set(A' \times B, C') \ar["\curry", r, swap] & \Set(A', \Set(B, C'))
  \end{tikzcd}\] for every pair of functions \(f : A' \to A\) and
\(g : C \to C'\). We calculate now: for every function
\(h : A \times B \to C\) we have
\[\curry \Set(f \times B, g) (h) = \curry \left(g h \left(f \times
      \id_B\right)\right) = \lambda x. \lambda y. gh(f(x), y) \] and
\[\begin{aligned}
  \Set(f, \Set(B, g)) \curry (h) &= \Set(f, \Set(B, g)) (\lambda x. \lambda y. h(x, y)) = \\
                                 &= (\lambda k. gk) (\lambda x. \lambda y. h(x, y)) f
\end{aligned}\]
hence they are equal. \NotaInterna{Since we happen to use lambda calculus quite often, write about it a little more.}
\end{example}

\begin{exercise}[Partial functions]
  For \(A\) and \(B\) sets, a {\em partial function} from \(A\) to
  \(B\) is relation \(f \subseteq A \times B\) with the property
  \begin{center}
    for every \(x \in A\) and \(y_1, y_2 \in B\), if
    \((x, y_1) \in f\) and \((x, y_2) \in f\) then \(y_1 = y_2\).
  \end{center}
  We want to compose partial functions as well: provided
  \(f \in \Par(A, B)\) and \(g \in \Par(B, C)\),
  \[gf \coloneq \set{(x, y) \in A \times C \mid (x, z) \in f \text{ and } (z, y) \in g
      \text{ for some } z \in B} .\] It is immediate to verify
  \(\Par\) complies the rules that make it a category. Indeed, this is
  the {\em category of partial functions}, written as \(\Par\): here,
  the objects are sets and the morphisms are partial functions.\newline
  Suppose given a partial function \(f : A \to B\). For every
  \(x \in A\) the possibilities are two: there is one element of \(B\)
  bound to is, and we write it \(f(x)\), or none. What if we
  considered {\em no value} as an output value?  Provided two sets
  \(A\) and \(B\) and a partial function \(f : A \to B\), we assign an
  actual function
  \[
    \bar f : A \to B+1 \,, \ \bar f(x) \coloneq
    \begin{cases} f(x) & \text{if \(x\) has an element of \(B\) bound}
      \\
      \ast & \text{otherwise}
    \end{cases}
  \]
  where \(1 \coloneq \set{\ast}\) with \(\ast\) designating the absence of
  output. It is quite simple to show that
  \[\Par(A,
    B) \to \Set(A, B+1)\,,\ f \to \bar f\] is a bijection for every couple
  of sets \(A\) and \(B\). Now it's up to you to categorify this: find
  two functors that make an adjunction
  \[\begin{tikzcd}
      \Set \ar["I"{name=I}, r, bend left] & \Par
      \ar["J"{name=J}, l, bend left] \ar["\perp"{description}, phantom,
      from=I, to=J]
    \end{tikzcd}\]
\end{exercise}

In the previous section, we have deduced from the notion of adjunction
from terminal and initial objects of certain comma categories. The
converse is true: adjunctions give terminal and initial objects in the
same comma categories.

\begin{figure}
  \centering
  \begin{tabular}{c}
    \input{adjointness/unit.qtikz} \\ 
    \midrule
    \input{adjointness/counit.qtikz}
  \end{tabular}
  \caption{Units and co-units of an adjunction}
\end{figure}

\begin{proposition}\label{proposition:UnitsAndCounits}
  Provided you have locally small categories and functors
  \[\begin{tikzcd}[column sep=small]
      \cat C \ar["L", r, shift left] & \cat D \ar["R", l, shift left]
    \end{tikzcd}\] and an adjunction
  \[\alpha : \cat D(L(\hole), \hole) \to \cat C(\hole, R(\hole)) .\]
  Then:
  \begin{tcbenum}
  \item For \(x \in \obj{\cat C}\) the morphisms
    \[\eta_x : x \to RL(x) \,,\ \eta_x := \alpha_{x,
        L(x)}\left(\id_{L(x)}\right)\] form a natural transformation
    \(\eta : \id_{\cat C} \tto RL\) such that \(\eta_x\) is initial in
    \(x {\downarrow} R\).
  \item For \(y \in \obj{\cat C}\) the morphisms
    \[\theta_y : LR(y) \to y \,,\ \theta_y := \inv\alpha_{R(y),y}
      \left(\id_{R(y)}\right)\] form a natural transformation
    \(\theta : LR \tto \id_{\cat D}\) such that \(\theta_y\) is terminal in
    \(L {\downarrow} y\).
  \end{tcbenum}
\end{proposition}

\begin{proof}\NotaInterna{Rewrite proof.}
  Let us write the adjunction of the statement above as
  \[\bar\hole : \cat D (L(\hole), \hole) \tto \cat C (\hole, R(\hole))
    .\] We verify that
  \[\begin{tikzcd}
      x \ar["{\eta_x}", r] \ar["f", d, swap] & RL(x) \ar["RL(f)", d] \\
      y \ar["{\eta_y}", r, swap] & RL(y)
    \end{tikzcd}\] commutes for every \(f\) in \(\cat C\). In fact,
  \begin{align*}
    & RL(f) \eta_x = RL(f) \bar{\id_{L(x)}} \id_x = \bar{L(f) \id_{L(x)} \id_{L(x)}} = \bar{L(f)} \\
    & \eta_y f = R(\id_{L(y)}) \bar{\id_{L(y)}} f = \bar{\id_{L(y)} \id_{L(y)} L(f)} = \bar{L(f)} .
  \end{align*}
  It remains to show that the morphisms \(\eta_x : x \to RL(x)\) are
  initial in \(x {\downarrow} R\). In \(\cat C\) we draw
  \[\begin{tikzcd}
      x \ar["{\eta_x}", r] \ar["g", dr, swap] & RL(x) \\
      & R(y)
    \end{tikzcd}\] We know that there is one and only one
  \(h : L(x) \to y\) such that \(g = \bar{h}\). Then
  \[g = \bar{h \id_{L(x)} L\left(\id_x\right)} = R(h) \bar{\id_{L(x)}}
    \id_{x} = R(h) \eta_x . \qedhere .\]
\end{proof}

\newcommand\uncurry{\operatorname{uncurry}}

\begin{example}[Unit and counit for \(\curry\)]
  Recover Example~\ref{example:CurryingFunctions} now, because we
  calculate the units and counits, in the exact manner of the
  Proposition~\ref{proposition:UnitsAndCounits}. Here, the left
  adjoint is \((\times B)\) and the right adjoint is
  \(\Set(B, \hole)\). Thus the unit relative to the set \(A\) is the
  natural transformation having as components the images of
  \(\id_{A \times B}\) under \(\curry\), that is
  \[\eta_A := \curry(\id_{A \times B}) = \lambda x. \lambda y. (x, y) .\]
  For the determination of the counit, we need the inverse of
  \(\curry\) first of all:
  \[
    \begin{aligned}
      & \uncurry : \Set(A, \Set(B, C)) \to \Set(A \times B, C) \\
      & \uncurry f := \lambda (x, y). f(x)(y)
    \end{aligned}
  \]
  (Spend some time to understand why this is true.) The counit
  relative to the set \(C\) is the natural transformation having as
  components the images of \(\id_{\Set(B, C)}\) under \(\uncurry\),
  that is
  \[\theta_C := \uncurry\left(\id_{\Set(B, C)}\right) = \lambda (f,
    x). f(x) .\] Don't forget that \(\eta\) and \(\theta\) bring universal
  properties! We write them out. \YetToBeTeXed
\end{example}

\begin{remark}
  The example above is interesting:
  Exercise~\ref{exercise:ExponentialObject} will ask you to be more
  general. Moreover, you have just gained a universal property for the
  powerset, haven't you? In the same section, you can find something
  in that direction and started in the chapter of
  limits. \YetToBeTeXed
\end{remark}

%%% Local Variables:
%%% mode: LaTeX
%%% TeX-master: "../CT"
%%% TeX-engine: luatex
%%% End:
