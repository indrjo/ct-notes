% !TEX program = lualatex

% General notations
\newcommand\nil\varnothing
\newcommand\set[1]{\left\{#1\right\}}
\newcommand\angled[1]{\left\langle#1\right\rangle}
\newcommand\abs[1]{\left\lvert#1\right\rvert}
\newcommand\Abs[1]{\left\lVert#1\right\rVert}
\renewcommand\bar\overline
\renewcommand\tilde\widetilde
\renewcommand\hat\widehat
\newcommand\tto\Rightarrow
\newcommand\lrarr\Leftrightarrow
\newcommand\naturals{\mathbb N}
\newcommand\integers{\mathbb Z}
\newcommand\reals{\mathbb R}
\newcommand\sph{\mathbb S}
\newcommand\dsc{\mathbb D}
\newcommand\rpspc{\mathbb{RP}}
\newcommand\inout[1]{\textstyle#1\displaystyle}
\newcommand\epsln\varepsilon

\newcommand\hole{\phantom\square}

% Notations pertaining Category Theory
\newcommand\cat\mathcal
\newcommand\obj[1]{\left\lvert#1\right\rvert}
\newcommand\functo[1]{\xrightarrow{\ #1\ }}
\newcommand\funcot[1]{\xleftarrow{\ #1\ }}
\newcommand\functto[1]{\xRightarrow{\ #1\ }}
\newcommand\id{\mathtt 1}
\newcommand\comp{\operatorname{comp}}
\newcommand\inv[1]{#1^{-1}}
\newcommand\op[1]{#1^\text{op}}
%\newcommand\op[1]{\reflectbox{\(#1\)}}
\newcommand\opcat[1]{\op{\cat #1}}
%\newcommand\opcat[1]{\reflectbox{\(\cat #1\)}}
\newcommand\ev{\operatorname{ev}}
\newcommand\sk{\operatorname{sk}}
\newcommand\proj{\operatorname{pr}}
\newcommand\inj{\operatorname{in}}
\newcommand\linj{\mathtt{left}}
\newcommand\rinj{\mathtt{right}}
\newcommand\precirc[2]{\operatorname{pre}_{#1, #2}}
\newcommand\postcirc[2]{\operatorname{post}_{#1, #2}}

\newcommand\Inj{\operatorname{Inj}}
\newcommand\im{\operatorname{im}}

\newcommand\subobjs{\operatorname{Sub}}

% Some categories
\newcommand\Rel{\mathbf{Rel}}
\newcommand\Set{\mathbf{Set}}
\newcommand\FinSet{\mathbf{FinSet}}
\newcommand\Eqv{\mathbf{Eqv}}
\newcommand\Par{\mathbf{Par}}
\newcommand\Grp{\mathbf{Grp}}
\newcommand\Ring{\mathbf{Ring}}
\newcommand\Top{\mathbf{Top}}
\newcommand\Mat{\mathbf{Mat}}
\newcommand\Modu{\mathbf{Mod}}
\newcommand\Vect{\mathbf{Vect}}
\newcommand\FDVect{\mathbf{FDVect}}
\newcommand\Dyn{\mathbf{Dyn}}
\newcommand\cn{\mathbf{Cn}}
\newcommand\cocn{\mathbf{CoCn}}

% A shortcut for rendering a natural transformation.
\newcommand\naturaltr[5]{
  % Just in case, better use an "ampersand replacement"...
  % Should I employ the option "cramped" too?
  \begin{tikzcd}[ampersand replacement=\&]
    #2 \ar["#4"{name=dom-cat}, r, bend left=35]
       \ar["#5"{name=cod-cat}, r, bend right=35, swap]
    \&
    #3 \ar["#1", from=dom-cat, to=cod-cat, natural] 
  \end{tikzcd}
}

\newcommand\adjoint[4]{
  \begin{tikzcd}[ampersand replacement=\&]
    #1 \ar["{#2}", r, shift left] \& #3 \ar["{#4}", l, shift left]
  \end{tikzcd}
}

% Render a hiragana yo. Have a look at: https://tex.stackexchange.com/a/602616
% !!! Make sure you have installed the package "cjk" (if you are using TeXlive).
%\DeclareFontFamily{U}{dmjhira}{}
%\DeclareFontShape{U}{dmjhira}{m}{n}{ <-> dmjhira }{}
%\DeclareRobustCommand{\yo}{\text{\usefont{U}{dmjhira}{m}{n}\symbol{'110}}}
\newcommand\yo{\mathcal{Y}}

% I'd like a \mathtt\lambda, but I don't know how to have this here.
\newcommand\lam\lambda

% This is a *temporary* command: I need it to make notes around
% the text. These notes shall be removed in the final document.
\newcommand\NotaInterna[1]{{\color{red} [#1]}}
\newcommand\YetToBeTeXed{\NotaInterna{Yet to be \TeX{}ed\dots{}}}
