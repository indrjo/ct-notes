
\section{The Yoneda Lemma}

There is nothing special about the \(\cat C(a', -)\) of the previous section. If we take a functor \(X : \cat C \to \Set\), we could readily see the construction there can be recycled nicely:
%
\begin{quotation}
If you are provided a natural transformation \(\eta : \cat C(a, -) \tto X\), then just consider \(\eta_a : \cat C(a, a) \to X(a)\), and pick the element \(\eta_a(\id_a)\) of \(X(a)\).
\end{quotation}
%
In the previous section, we managed to prove that \(\eta_a(\id_a)\) is an iso, but that happened in a context where we had a natural {\em isomorphism}. What we will do here is realising that there is some bijection. More precisely:

\begin{lemma}\label{lemma:YonedaLemmaLemma}
Let \(\cat C\) be a locally small category, \(a \in \obj{\cat C}\) and functor \(X : \cat C \to \Set\). Then
\[[\cat C, \Set](\cat C(a, -), X) \cong X(a) .\]
\end{lemma}

%In particular, the classes \([\cat C, \Set](\cat C(a, -), X)\) are  actual sets.

\begin{proof}
The function we need is already isolated, we just give it name that cares with pedices because all in this proof will be reused later:
%
\[\begin{aligned}
& \theta_{a, X} : [\cat C, \Set](\cat C(a, -), X) \to X(a) \\
& \theta_{a, X}(\eta) := \eta_a(\id_a) .
\end{aligned}\]
%
We have the peieces of a puzzle here: we have some \(x \in X(a)\) and wish to contruct an appropriate natural tranformation \(\eta : \cat C(a, -) \tto X\), that is a family of functions \(\eta_b : \cat C(a, b) \to X(b)\) for \(b \in \obj{\cat C}\). Well, a morphism \(f : a \to b\) can be upgraded to a function \(X(f) : X(a) \to X(b)\) and, if we must land somewhere onto \(X(b)\), then we could just apply \(X(f)\) to that \(x\). In short, from a single element \(x \in X(a)\) we have a collection of functions
\[\lambda f . X(f)(x) : \cat C(a, b) \to X(b) \quad \text{for } b \in \obj{\cat C} .\]
Now, we have to verify if these function form a natural transformation as \(b\) varies. Take any morphism \(h : b \to c\) in \(\cat C\) and then consider the diagram
\[\begin{tikzcd}[column sep=large]
  \cat C(a, b) \ar["{\lambda m . hm}", d, swap] \ar["{\lambda f . X(f)(x)}", r] & X(b) \ar["{X(h)}", d] \\
\cat C(a, c) \ar["{\lambda f . X(f)(x)}", r, swap] & X(c)
\end{tikzcd}\]
We can immediately see the diagram commutes, hence what we have construed is a genuine natural tranformation. At this point we have a function
\[\begin{aligned}
  X(a) &\to [\cat C, \Set](\cat C(a, -), X) \\
  x    &\to \lambda f . X(f)(x)
\end{aligned}\]
It only remains to veirify the two functions here are one the inverse of the another. Provided \(x \in X(a)\), you have the natural transformation \(\lambda f . X(f)(x) : \cat C(a, -) \tto X\). Now, just apply the component \(\lambda f . X(f)(x) : \cat C(a, a) \to X(a)\) to \(\id_a\). The output is
\[X(\id_a)(x) = \id_{X(a)} (x) = x .\] 
Take a natural transformation \(\eta : \cat C(a, -) \tto X\) and consider \(\eta_a (\id_a)\). This element of \(X(a)\) will bring you to the natural tranformation \(\lambda f . X(f) (\eta_a(\id_a))\). Remember that \(\eta\) is a natural transformation \(\cat C (a, -) \tto X\), thus, if \(f : a \to b\), we have
\[\lambda f . X(f)(\eta_a(\id_a)) = \lambda f . \eta_b(f\id_a) = \lambda f . \eta_b (f) = \eta_b .\]
It works!
\end{proof}

\NotaInterna{What follows has to be rewritten.}

We have the {\em evaluation functor}
%\[\begin{aligned}
%& \ev_{\cat C} : \cat C \times [\cat C, \Set] \to \Set \\
%& \ev_{\cat C}(x, F) := F(x) \\
%& \ev\left(\begin{tikzcd}[row sep=small] a \ar["f", d, swap] \\ b \end{tikzcd}, \begin{tikzcd}[row sep=small] F \ar["\eta", d, swap, Rightarrow] \\ G \end{tikzcd}\right) := \eta_b F(f) = G(f) \eta_a . \\
%\end{aligned}\]
\[\ev_{\cat C} : \cat C \times [\cat C, \Set] \to \Set\]
that on objects
\[\ev_{\cat C}(x, F) := F(x)\]
and on morphisms
\[\ev\left(\begin{tikzcd}[row sep=small] a \ar["f", d] \\ b \end{tikzcd}, \begin{tikzcd}[row sep=small] F \ar["\eta", d, swap, Rightarrow] \\ G \end{tikzcd}\right) := \eta_b F(f) = G(f) \eta_a .\]

%\begin{lemma}[A lemma for the Yoneda Lemma]\label{lemma:YonedaLemmaLemma}
%Let \(\cat C\) be a locally small category. Then for every \(x \in \obj{\cat C}\) and functor \(F : \cat C \to \Set\),
%\[[\cat C, \Set](\cat C(x, -), F) \cong F(x) .\]
%In particular, the classes \([\cat C, \Set](\cat C(x, -), F)\) are  actual sets.
%\end{lemma}

%\begin{proof}
%For \(x\) and \(F\) as in the hypothesis, take functions
%%\begin{gather*}
%\[\lambda_{x, F} : [\cat C, \Set](\cat C(x, -), F) \to F(x)\,,\ \lambda_{x, F}(\alpha) := \alpha_x(\id_x) .\]
%%\end{gather*}
%Now, for every \(a \in F(x)\) we have the transformation \(\mu_{x, F}(a)\) from \(\cat C(x, \bullet)\) to \(F\) which has the components
%\[\cat C (x, c) \to F(c)\,,\ f \to \big(F(f)\big)(a) ;\]
%it is immediate to show that it is natural. Thus we have functions
%\[\mu_{x, F} : F(x) \to [\cat C, \Set](\cat C(x, -), F) .\]
%We prove
%\[\begin{aligned}
%& \lambda_{x, F} \mu_{x, F} = \id_{F(x)} \\
%& \mu_{x, F} \lambda_{x, F} = \id_{[\cat C, \Set](\cat C(x, -), F)}.
%\end{aligned}\]
%In fact, for \(a \in F(x)\) we have \(\lambda_{x, F} \big(\mu_{x, F} (a)\big)\) is the component \(\cat C(x, x) \to F(x)\) of \(\mu_{x, F}(a)\) evaluated at \(\id_x\), viz \(\id_{F(x)}(a) = a\). Besides, for if \(\alpha : \cat C(x, \bullet) \tto F\) natural transformation we have \(\mu_{x, F}\big(\lambda_{x, F} (\alpha)\big) = \mu_{x, F}\big(\alpha_x (\id_x)\big)\) is the natural transformation \(\cat C(x, \bullet) \tto F\) with components
%\[\cat C(x, c) \to F(c)\,,\ f \to \big(F(f)\big)(\alpha_x(\id_x)) = \alpha_c(f)\]
%for \(c \in \obj{\cat C}\); that is \(\mu_{x, F} \lambda_{x, F} (\alpha) = \alpha\). The proof is complete now.
%\end{proof}

Let \(\cat C\) be a locally small category.
We have the functor
\[\yo_{\cat C} : \cat C \times [\cat C, \Set] \to \Set\]
given on objects as follows
\[\yo_{\cat C}(x, F) := [\cat C, \Set]\big(\cat C(x, -), F\big)\]
and on morphisms
\[\left[\yo_{\cat C}\left(\begin{tikzcd}[row sep=small] a \ar["f", d, swap] \\ b \end{tikzcd}, \begin{tikzcd}[row sep=small] F \ar["\eta", d, swap, Rightarrow] \\ G \end{tikzcd}\right)\right]\left(\begin{tikzcd}[cramped, row sep=small] {\cat C(a, -)} \ar["\alpha", d, Rightarrow] \\ F \end{tikzcd}\right) :=
\set{\left. \cat C(b, c) \functo{\eta_c \alpha_c (\_ f)} G(c) \right\mid c \in \obj{\cat C}}.\]
Observe that Lemma~\ref{lemma:YonedaLemmaLemma} solves annoying size issues in the definition of \(\yo_{\cat C}\) on objects. While the statement of this lemma is important for technical reasons, its proof guides us to the following completion.

\begin{proposition}[Yoneda Lemma]
For \(\cat C\) locally small category, \(\yo_{\cat C} \cong \ev_{\cat C}\).
\end{proposition}

\begin{proof}
The transformation \(\lambda : \yo_{\cat C} \tto \ev_{\cat C}\) having as components the functions \(\lambda_{x, F}\) of the proof of Lemma~\ref{lemma:YonedaLemmaLemma} is natural, that is
\[\begin{tikzcd}
\yo_{\cat C} (a, F) \ar["\lambda_{a, F}", r] \ar["{\yo_{\cat C}(f, \eta)}", d, swap] & \ev_{\cat C} (a, F) \ar["{\ev_{\cat C}(f, \eta)}", d] \\
\yo_{\cat C} (b, G) \ar["\lambda_{b, G}", r, swap] & \ev_{\cat C}(b, G)
\end{tikzcd}\]
commutes for every \(f \in \cat C(a, b)\) and \(\eta \in [\cat C, \Set](F, G)\). In fact, for every natural transformation \(\eta \in \yo_{\cat C}(a, F)\) we have
\[\ev_{\cat C}(f, \eta) \big(\lambda_{a, F} (\alpha)\big) =  \eta_b\alpha_b \cat C(a, f)(\id_a) = \eta_b \alpha_b f ;\]
besides,
\[\lambda_{b, G} \big(\yo_{\cat C}(f, \eta) (\alpha)\big) = \eta_b \alpha_b (\_ f)(1_b) = \eta_b \alpha_b f .\]
%
We can conclude \(\lambda\) is an isomorphism, as the proof of Lemma~\ref{lemma:YonedaLemmaLemma} tells us its components are isomorphisms.
\end{proof}

%%% Local Variables:
%%% mode: LaTeX
%%% TeX-master: "../CT"
%%% TeX-engine: luatex
%%% End:
