% !TEX program = lualatex
% !TEX root = ./CT.tex 

\documentclass[ structure      = book
              , maketitlestyle = standard
              , partfont       = smallcaps
              , partstyle      = parright 
              , chapstyle      = parcenter
              , secstyle       = center
              , secfont        = roman
              , subsecstyle    = center
              , subsecfont     = roman
              , twocolcontents = toc
              , liststyle      = aligned
              , quotesize      = normalsize
              , headerstyle    = center
              , footnotestyle  = dotted
              ]{suftesi}

\captionsetup{ width=.8\textwidth
             , labelfont=bf
             }

\newenvironment{tcbenum}
  {\begin{enumerate}[leftmargin=*]}
  {\end{enumerate}}

\newenvironment{tcbitem}
  {\begin{itemize}[leftmargin=*]}
  {\end{itemize}}

\usepackage[no-math]{fontspec}
\usepackage[rm,tt=false]{libertine}
\usepackage[scale=.85]{sourcecodepro}

\usepackage{polyglossia}
\setmainlanguage[variant=uk]{english}
\setotherlanguage{german}

\usepackage[english=british]{csquotes}
\newcommand\q\enquote

\usepackage{hyperref}
\hypersetup{breaklinks,hidelinks}

\usepackage{anyfontsize}

\usepackage{indentfirst}

\usepackage{booktabs}

\usepackage[
           %  bibstyle=authoryear
           %, citestyle=authortitle-icomp
             bibstyle = alphabetic
           , citestyle = alphabetic
           , pluralothers=true
           , autolang=langname
           ]{biblatex}
\addbibresource{biblio.bib}
\nocite{*} % temporary

\usepackage{libertinust1math}
\usepackage{MnSymbol}
\usepackage{mathtools}
\let\underbrace\LaTeXunderbrace
\let\overbrace\LaTeXoverbrace
\usepackage[bb=stix]{mathalfa}

\usepackage{amsthm}
\newcounter{coun} % Find a nicer name for the counter...
\counterwithin{coun}{section}
\theoremstyle{definition}
\newtheorem{definition}  [coun]{Definition}
\newtheorem{construction}[coun]{Construction}
\newtheorem{lemma}       [coun]{Lemma}
\newtheorem{proposition} [coun]{Proposition}
\newtheorem{corollary}   [coun]{Corollary}
\newtheorem{remark}      [coun]{Remark}
\newtheorem{sandbox}     [coun]{Sandbox}
\newtheorem{example}     [coun]{Example}
\newtheorem{exercise}    [coun]{Exercise}

\renewcommand\theequation{\thesection.\arabic{equation}}

\newcommand\inlinethm[1]{\stepcounter{coun}{\bf #1~\thecoun}}

%\usepackage[skins,theorems,breakable]{tcolorbox}
%\tcbset{ enhanced
%       , breakable
%       , blanker
%       }
%
%\tcolorboxenvironment{definition}
%  { tile
%  , top=6pt
%  , bottom=6pt
%  , left=6pt
%  , right=6pt
%  , borderline={1pt}{0pt}{gray!65!black}
%  , colback=gray!15!white
%  }
%\tcolorboxenvironment{lemma}
%  { left=6pt
%  , borderline west={.75mm}{0mm}{blue!65!black}
%  }
%\tcolorboxenvironment{proposition}
%  { left=6pt
%  , borderline west={.75mm}{0mm}{blue!65!black}
%  }
%\tcolorboxenvironment{corollary}
%  { left=6pt
%  , borderline west={.75mm}{0mm}{blue!65!black}
%  }
%  
%\tcolorboxenvironment{example}
%  { top=6pt
%  , bottom=6pt
%  , borderline north={1pt}{0mm}{gray!65!black}
%  , borderline south={1pt}{0mm}{gray!65!black}
%  }

\usepackage{adjustbox}

\usepackage{tikz}
\usetikzlibrary{calc,babel,arrows.meta}
\tikzset{>={To[length=3pt,width=3pt]}}
\usepackage{tikz-cd}
\tikzcdset{
          , arrow style=tikz
          , diagrams={>={To[length=3pt,width=3pt]}}
          , shorten=-2pt
          , natural/.style={shorten=2pt,Rightarrow}
          %, cramped
          }

% The first chapter is the zeroth chapter.
\setcounter{chapter}{-1}

