
\section{The hom functor}

In a locally small \NotaInterna{have we defined somewhere that?}
category \(\cat C\), take two morphisms
\[\begin{tikzcd}
    a_1 \\
    b_1 \ar["g_1", u]
  \end{tikzcd} \quad\text{and}\quad \begin{tikzcd}
    a_2 \ar["g_2", d] \\
    b_2
  \end{tikzcd}\] For \(f : a_1 \to a_2\) we have the composite
\(g_2 f g_1 : b_1 \to b_2\), that is a function
\[\cat C (a_1, a_2) \to \cat C (b_1, b_2) .\]
We will refer to this function using lambda calculus notation:
\(\lambda f . g_2 f g_1\). If we have
\[\begin{tikzcd}%[row sep=small]
    a_1 \ar["f", r] & a_2 \ar["g_2", d] \\
    b_1 \ar["g_1", u] & b_2 \ar["h_2", d] \\
    c_1 \ar["h_1", u] & c_2
  \end{tikzcd}\] we can say that
\[\lambda f . (h_2 g_2) f (g_1 h_1) = (\lambda f' . h_2 f' h_1) (\lambda f . g_2 f g_1)
  ,\] which can be derived by uniquely using the associativity of the
composition. Another remarkable property can be obtained when
\(a_1 = b_1\), \(g_1 = \id_{a_1}\), \(a_2 = b_2\) and
\(g_2 = \id_{a_2}\):
\[\lambda f . \id_{a_2} f \id_{a_1} = \lambda f . f = \id_{\cat C (a_1, b_1)} .\]
% that is \(f\) is mapped to the identity function on
% \(\cat C (a_1, b_1)\).
There is functoriality, to understand that, we need to package all
this machinery in one functor. The functor we are looking for is
\[\hom_{\cat C} : \opcat C \times \cat C \to \Set\]
which takes every \((x, y) \in \obj{\opcat C \times \cat C}\) to
\(\cat C (x, y)\) and
\[\hom_{\cat C} \left(\begin{tikzcd}[row sep=small] a_1 \\ b_1
      \ar["g_1", u] \end{tikzcd}, \begin{tikzcd}[row sep=small] a_2
      \ar["g_2", d] \\ b_2 \end{tikzcd}\right)
  \left(\begin{tikzcd}[column sep=small] a_1 \ar["f", r] &
      a_2 \end{tikzcd}\right) := g_2 f g_1 .\]


%%% Local Variables:
%%% mode: LaTeX
%%% TeX-master: "../CT"
%%% TeX-engine: luatex
%%% End:
