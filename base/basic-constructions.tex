
\section{Basic constructions}

In this section, we will present the first and most basic
constructions involving categories.

For \(\cat C\) a category, its {\em dual} (or {\em opposite}) category
is denoted \(\op{\cat C}\) and is described as follows. Here, the
objects are the same of \(\cat C\) and \q{being a morphism
  \(a \to b\)} exactly means \q{being a morphism \(b \to a\) in
  \(\cat C\)}. In other words, passing from a category to its dual
leaves the objects unchanged, whereas the morphisms have their verses
reversed. To dispel any ambiguity, by \q{reversing} the morphisms we
mean that morphisms \(f : a \to b\) of \(\cat C\) can be found among the
morphisms \(b \to a\) of \(\op{\cat C}\) and, vice versa, morphisms
\(a \to b\) of \(\op{\cat C}\) among the morphisms \(b \to a\) of
\(\cat C\). Nothing is actually constructed out of the blue. Some
authors suggest to write \(\op f\) to indicate that one \(f\) once it
has domain and codomain interchanged, but we do not do that here,
because they really are the same thing but in different places. So, if
\(f\) is the name of a morphism of \(\cat C\), the name \(f\) is kept
to indicate that morphism as a morphism of \(\op{\cat C}\); obviously,
the same convention applies in the opposite direction. It may seem we
are going to nowhere, but it makes sense when it comes to define the
compositions in \(\op{\cat C}\): for \(f : a \to b\) and
\(g : b \to c\) morphsisms of \(\op{\cat C}\) the composite arrow is so
defined
\[gf := fg .\label{defeqn:DualComp}\] This is not a commutative
property, though. Such definition is to be read as follows. At the
left side, \(f\) and \(g\) are to be intended as morphisms of
\(\op{\cat C}\) that are to be composed therein. Then the composite
\(gf\) is calculated as follows:
\begin{tcbenum}
\item look at \(f\) and \(g\) as morphisms of \(\cat C\) and compose
  them as such: so \(f : b \to a\) and \(g : c \to b\) and
  \(fg : c \to a\) according to \(\cat C\);
\item now regard \(fg\) as a morphism of \(\op{\cat C}\): this is the
  value \(gf\) is bound to.
\end{tcbenum}
Let us see now whether the categorial axioms are respected. For \(x\)
object of \(\op {\cat C}\) there is \(\id_x\), which is a morphism
\(x \to x\) in either of \(\cat C\) and \(\op{\cat C}\). For every
object \(y\) and morphism \(f : y \to x\) of \(\op{\cat C}\) we have
\[\id_x f = f \id_x = f .\]
Similarly, we have that
\[g \id_x = g\] for every object \(z\) and morphism \(g : x \to z\) of
\(\op{\cat C}\). Hence, \(\id_x\) is an identity morphism in
\(\op{\cat C}\) too. Consider now four objects and morphisms of
\(\op{\cat C}\)
\[a \functo f b \functo g c \functo h d\] and let us parse the
composition
\[h(gf) .\] In \(h(gf)\) regard both \(h\) and \(gf\) as morphisms of
\(\cat C\). In that case, \(h(gf)\) is exactly \((gf)h\), where \(gf\)
is \(fg\) once \(f\) and \(g\) are taken as morphisms of \(\cat C\)
and composed there. So \(h(gf) = (fg)h\), where at left hand side
compositions are performed in \(\cat C\): being the composition is
associative, \(h(gf) = (fg)h = f(gh)\). We go back to \(\op{\cat C}\),
namely \(f(gh)\) becomes \((gh)f\) and \(gh\) becomes \(hg\), so that
we eventually get the associativity
\[h(gf) = (hg) f .\]

It may seem hard to believe, but duality is one of the biggest
conquest of Category Theory. \NotaInterna{Talk more about duality
  here\dots{}} This construction may seem a useless sophistication for
now, but later we will discover how this serves the scope to make
functors encompass a broader class of constructions. However, as for
now, let us see all this under a new light: what does duality mean for
diagrammatic reasoning? Commuting triangles
\[\begin{tikzcd}[row sep=small]
    & b \ar["g", dd] \\
    a \ar["f", ur] \ar["h", dr, swap] \\
    & c
  \end{tikzcd}\] of \(\cat C\) are exactly commuting triangles
\[\begin{tikzcd}[row sep=small]
    & b \ar["f", dl, swap] \\
    a \\
    & c \ar["h", ul] \ar["g", uu, swap]
  \end{tikzcd}\] in \(\opcat C\).

\begin{exercise}
  In this way, it should be even more immediate to prove the two
  categorial axioms. Give it a try. Observe this approach is the mere
  translation of what we have conveyed with words above.
\end{exercise}

\begin{example}[Dual prosets]
  We already know how prosets are categories; let \((P, \leqslant)\) be one of
  them. Here the morphisms are exactly the pairs \((b, a)\) such that
  \((a, b) \in \leqslant\). If we rephrase all this, we can introduce the dual
  relation \(\geqslant\) defined by: \(b \geqslant a\) if and only if \(a \geqslant b\).
\end{example}

\begin{exercise}
  Consider a single object category \(\cat G\), that is a monoid. What
  is \(\opcat G\)? What is \(\opcat G\) if \(\cat G\) is a group?
\end{exercise}

The concept of duality for categories has one important consequence on
statements written in a \q{categorial language}. We do not need to be
fully precise here: they are statements written in a sensible way
using the usual logical connectives, names for objects, names for
morphisms and quantifiers acting on such names.

\begin{example}
  If we have a morphism \(f : a \to b\) in some category \(\cat C\),
  consider the statement
  \begin{quotation}
    For every {\color{red!65!black} object \(c\) of \(\cat C\)} and
    {\color{blue!65!black} morphisms \(g_1, g_2 : c \to a\) in
      \(\cat C\)}, if {\color{green!65!black} \(f g_1 = f g_2\)} then
    {\color{orange!65!black}\(g_1 = g_2\)}.
  \end{quotation}
  If you remember, it is just said that \(f\) is a monomorphism. We
  operate a translation that doesn't modify the truth of the sentence:
  that is, if it is true, it remains so; it is false, it remains
  false.
  \begin{quotation}
    For every {\color{red!65!black} object \(c\) of \(\opcat C\)} and
    {\color{blue!65!black} morphisms \(g_1, g_2 : a \to c\) in
      \(\opcat C\)}, if {\color{green!65!black} \(g_1 f = g_2 f\)}
    then {\color{orange!65!black}\(g_1 = g_2\)}.
  \end{quotation}
  If we regard \(f\) as a morphism \(b \to a\) of \(\opcat C\), then
  \(f\) is an epimorphism in \(\opcat C\).
\end{example}

Let us try to settle this explicitly: if we have a categorial
statement \(p\), the dual of \(p\) --- we may call \(\op{p}\) --- is the
statement obtained from \(p\) keeping the connectives and the
quantifiers of \(p\), whereas the other parts are replaced by their
dual counterparts.

\begin{example}
  \NotaInterna{Anticipate products and co-products\dots{}}
\end{example}

Another useful construction is that of product of categories. Assuming
we have two categories \(\cat C_1\) and \(\cat C_2\), the product
\(\cat C_1 \times \cat C_2\) is the category in which
\begin{tcbitem}
\item The objects are the pairs
  \((a, b) \in \obj{\cat C_1} \times \obj{\cat C_2}\).
\item Being a morphism \((a_1, a_2) \to (b_1, b_2)\) means being a
  pair of two morphisms
  \[\left(\begin{tikzcd}[row sep=small] a_1 \ar["{f_1}", d, swap] \\
        b_1 \end{tikzcd}, \begin{tikzcd}[row sep=small] a_2
        \ar["{f_2}", d] \\ b_2 \end{tikzcd}\right)\] where \(f_1\) is
  in \(\cat C_1\) and \(f_2\) in \(\cat C_2\); we write such morphism
  as \(\left(f_1, f_2\right)\).
\item The composition is defined component-wise
  \[\left(\begin{tikzcd}[row sep=small] b_1 \ar["{g_1}", d, swap] \\
        c_1 \end{tikzcd}, \begin{tikzcd}[row sep=small] b_2
        \ar["{g_2}", d] \\ c_2 \end{tikzcd}\right)
    \left(\begin{tikzcd}[row sep=small] a_1 \ar["{f_1}", d, swap] \\
        b_1 \end{tikzcd}, \begin{tikzcd}[row sep=small] a_2
        \ar["{f_2}", d] \\ b_2 \end{tikzcd}\right) := (g_1 f_1, g_2
    f_2) .\]
\end{tcbitem}

\begin{exercise}
  Verify categorial axioms hold for the product of two categories.
\end{exercise}

In future, we will need to consider product categories of the form
\(\opcat C \times \cat D\). For objects, there is nothing weird to
say; about morphism, observe that a morphism
\[\begin{tikzcd} (a, b) \ar["{(f, g)}", d] \\ (a', b')\end{tikzcd}\]
is precisely the pair
\[\left(\begin{tikzcd}[row sep=small] a \\ a' \ar["f",
      u] \end{tikzcd}, \begin{tikzcd}[row sep=small] b \ar["g", d] \\
      b \end{tikzcd}\right)\] whose first component comes from
\(\cat C\) while the second one from \(\cat D\).

Let as now talk about comma categories.

\begin{example}
  Words are labels humans attach to things to refer to them. Different
  groups of speakers have developed different names for the
  surrounding world, which resulted in different languages. We can use
  sets to gather the words present in any language. Now, if we are
  given a set \(\Omega\) of things and a set \(L\) of the words of a
  chosen language, then a function \(\lambda : \Omega \to L\) can be
  seen as the act of labelling things with names.\footnote{Of course,
    our discourse is rather simplified here: everything in \(\Omega\)
    has one and only one dedicated word, which is not always the
    case. In fact, the existence of synonyms within a language
    undermines the requirement of uniqueness. Further, a language
    might not have words for everything: for instance, German has the
    word \textgerman{Schilderwald}, which has not a corresponding
    single word in English --- if you want to explain the meaning it
    bears, you can say it is \q{a street that is so overcrowded and
      rammed with so many street signs that you are getting lost
      rather than finding your way.}} We will call such functions as
  {\em vocabularies} for \(\Omega\), although this might not be the
  official name.\newline Languages do not live in isolation with
  others: if we know how to translate words, we can understand what
  speakers of other languages are saying. Imagine now you have two
  vocabularies
  \[\begin{tikzcd}[column sep=small]
      A & & B \\
      & \Omega \ar["\alpha", ul] \ar["\beta", ur, swap]
    \end{tikzcd}\] To illustrate concept, think some
  \(\omega \in \Omega\): if \(\omega \in \Omega\) has the name
  \(\alpha(\omega)\) under the vocabulary \(\alpha\) and \(\omega\) is
  called \(\beta(\omega)\) according to \(\beta\), then a translation
  would be a correspondence between \(\alpha(\omega)\) and
  \(\beta(\omega)\). We could say, a {\em translation} from \(\alpha\)
  to \(\beta\) is a function \(\tau : A \to B\) such that
  \[\begin{tikzcd}[column sep=small]
      A \ar["\tau", rr] & & B \\
      & \Omega \ar["\alpha", ul] \ar["\beta", ur, swap]
    \end{tikzcd}\] commutes. As you may expect, translations can be
  composed to obtain translations: if we have two translations
  \(\eta\) and \(\theta\) as in the diagram
  \[\begin{tikzcd}
      & B \ar["\theta", dr] & \\
      A \ar["\eta", ur] & \Omega \ar["\alpha", l] \ar["\beta", u]
      \ar["\gamma", r, swap] & C
    \end{tikzcd}\] with the two triangles commuting, we have also the
  commuting
  \[\begin{tikzcd}[column sep=small]
      A \ar["\theta\eta", rr] & & C \\
      & \Omega \ar["\alpha", ul] \ar["\gamma", ur, swap]
    \end{tikzcd}\] This is interesting if we look at things with
  categories: objects are functions that have a domain in common, and
  we have selected as morphisms the functions between the codomains
  that make certain triangles commute. Other examples in such spirit
  follow.
\end{example}

% \begin{example}
%   Consider the set of the of the subsets of a given set \(X\), what
%   is commonly named \q{set of the parts} of \(X\) and denoted
%   \(2^X\). That set is (more than) a proset, in which case we have
%   relations \(A \subseteq X\) for \(A \in 2^X\). If we regard all
%   this categorially, we have morphisms \(A \to X\), one for each
%   \(A \in 2^X\). Observe the expression
%   \begin{quotation}
%     \(A, B \subseteq X\) and \(A \subseteq B\)
%   \end{quotation}
%   can be be rephrased as the commutativity of the triangle
%   \[\begin{tikzcd}[column sep=small]
%       A \ar[dr] \ar[rr] & & B \ar[dl] \\
%       & X
%     \end{tikzcd}\] (By the way, in prosets triangles always
%   commute.) Now if we have two commuting triangles
%   \[\begin{tikzcd}
%       & B \ar[d] \ar[dr, blue] \\
%       A \ar[r] \ar[ur, red] & X & C \ar[l]
%     \end{tikzcd}\] we obtain a commuting triangle
%   \[\begin{tikzcd}[column sep=small]
%       A \ar[dr, purple] \ar[rr] & & C \ar[dl] \\
%       & X
%     \end{tikzcd}\] composing the coloured morphisms. Indeed, there
%   is category in which morphisms are inclusions \(A \to X\),
%   morphisms from \(A \to X\) to \(B \to X\) are morphisms
%   \(A \to B\) (in this case, there is only one of such) giving
%   commuting triangles and compositions happen as just illustrated.
% \end{example}

\begin{example}[Covering spaces]
  Probably you have heard of covering spaces when you had to calculate
  the first homotopy group of \(\mathbb S^1\). More generally, a {\em
    covering space} of a topological space \(X\) is continuous
  function \(p : \tilde X \to X\) with the following property: there
  is a open cover \(\set{U_i \mid i \in I}\) of \(X\) such that every
  \(\inv p U_i\) is the disjoint union of a family
  \(\set{V_j^i \mid j \in J}\) of open subsets of \(\tilde X\) and the
  restriction of \(p\) to \(V_j^i\) is a homeomorphism
  \(V_j^i \to U_i\) for every \(i \in I\) and \(j \in
  J\). \NotaInterna{To be continued\dots{}}
\end{example}

\begin{example}[Field extensions]
  \NotaInterna{Yet to be \TeX{}-ed\dots{}}
\end{example}

%%% Local Variables:
%%% mode: LaTeX
%%% TeX-master: "../CT"
%%% TeX-engine: luatex
%%% End:
