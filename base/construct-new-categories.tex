% !TEX program = lualatex
% !TEX spellcheck = en_GB
% !TEX root = ../base.tex

\section{Construct new categories}

In this section, we will present some basic constructions yielding categories from other categories.

As from two sets you can obtain their product, we have something akin to that for two categories. For \(\cat C_1\) and \(\cat C_2\) categories, the product \(\cat C_1 \times \cat C_2\) is the category in which
\begin{tcbitem}
\item The objects are the pairs \((a, b) \in \obj{\cat C_1} \times \obj{\cat C_2}\), formally
\[\obj{\cat C_1 \times \cat C_2} := \obj{\cat C_1} \times \obj{\cat C_2}.\]
\item Being a morphism \((a_1, b_1) \to (a_2, b_2)\) means being a pair of two morphisms
\[\left(\begin{tikzcd}[row sep=small] a_1 \ar["f", d, swap] \\ a_2 \end{tikzcd}, \begin{tikzcd}[row sep=small] b_1 \ar["g", d] \\ b_2 \end{tikzcd}\right)\]
where \(f\) is in \(\cat C_1\) and \(g\) in \(\cat C_2\); we write such morphism as \(\left(f, g\right)\).
\item The composition is defined component-wise
\[\left(\begin{tikzcd}[row sep=small] a_2 \ar["f_2", d, swap] \\ a_3 \end{tikzcd}, \begin{tikzcd}[row sep=small] b_2 \ar["g_2", d] \\ b_3 \end{tikzcd}\right) \left(\begin{tikzcd}[row sep=small] a_1 \ar["f_1", d, swap] \\ a_2 \end{tikzcd}, \begin{tikzcd}[row sep=small] b_1 \ar["g_1", d] \\ b_2 \end{tikzcd}\right) := (f_2 f_1, g_2 g_1) .\]
\end{tcbitem}

\begin{exercise}
Verify categorial axioms hold for the product of two categories.
\end{exercise}

In future, we will need to consider product categories of the form \(\opcat C \times \cat D\). For objects, there is nothing to say. About morphism, keep in mind that a morphism
\[\begin{tikzcd} (a, b) \ar["{(f, g)}", d] \\ (a', b')\end{tikzcd}\]
is just the pair
\[\left(\begin{tikzcd}[row sep=small] a \\ a' \ar["f", u] \end{tikzcd}, \begin{tikzcd}[row sep=small] b \ar["g", d] \\ b \end{tikzcd}\right)\]
whose first component comes from \(\cat C\) while the second one from \(\cat D\).
